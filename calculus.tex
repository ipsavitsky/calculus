\documentclass[10pt]{article}
\pdfoutput=1
\usepackage[T2A]{fontenc}
\usepackage[utf8]{inputenc}
\usepackage[english, russian]{babel}
\usepackage{NotesTeX_rus}

\begin{document}
    \tableofcontents
    \section{Определения}
    \begin{definition}
        Вектор $P = (x_0, \ldots, x_n)$ называется \textbf{разбиением(неразмеченным)} отрезка $[a, b]$, если $a = x_0 < x_1 < \ldots < x_n = b$, $i$-й отрезок разбиения обозначаем $\Delta_i := [x_{i-1}, x_i]$  
    \end{definition}
    \begin{definition}
        Разбиение $P'$ отрезка $[a, b]$ называется \textbf{измельчением} разбиения $P$ того же отрезка, если все элементы $P$ так же являются элементами $P'$
    \end{definition}
    \begin{definition}
        Разбиение $R$ называется \textbf{объединением разбиений} $P$ и $Q$ отрезка $[a, b]$, если $R$ включает в себя все элементы $P$ и $Q$ и только их. Обозначение: $R = P \cup Q$
    \end{definition}
    \begin{definition}
        Пусть функция $f$ определена на отрезке $[a, b]$, $P_{\xi}$ - его размеченное разбиение. Тогда число
        $$
        \sigma(f, P, \xi) = \sigma(f, P_{\xi}) = \sigma(P_{\xi}) := \sum_{i=1}^n f(\xi_i)\Delta x_i
        $$
        где $\Delta x_i = |\Delta_i| := x_i - x_{i-1}$ и $n = n(P)$, называется \textbf{интегральной суммой} функции $f$, соответствующей размеченному разбиению $P_{\xi}$
    \end{definition}
    \begin{definition}
        Пусть $P$ - разбиение отрезка $[a, b]$. Тогда числа
        \begin{gather*}
            S_{*}(P) = \underline{S}(P) := \sum_{i = 1}^{n(P)} m_i \Delta x_i \\
            S^{*}(P) = \overline{S}(P) := \sum_{i = 1}^{n(P)}M_i \Delta x_i\\
            \text{где }m_k := \inf_{x \in \Delta_k}f(x);\, M_k := \sup_{x \in \Delta_k}f(x)
        \end{gather*}
        называются \textbf{нижней} и \textbf{верхней суммами Дарбу} функции $F$ для разбиения $P$ 
    \end{definition}
    % TODO Определение предела интегральных сумм при стремлении диаметра разбиения к нулю на языке \varepsilon-\delta
    \section{Билеты}
    \subsection{Применение дифференциального исчисления для построения графиков}
    \subsubsection{Определение: строгого локального экстремума функции одной переменной. Теорема: первое достаточное условие экстремума (с доказательством).}
    \subsubsection{Определение: строгого локального экстремума функции одной переменной. Теорема: второе достаточное условие экстремума (с доказательством).}
    \subsubsection{Определение: строгого локального экстремума функции одной переменной. Теорема: третье достаточное условие экстремума (с доказательством).}
    \subsubsection{Определение: график функции на данном интервале имеет выпуклость, направленную вверх (вниз). Теорема о связи между знаком второй производной и выпуклостью (с доказательством).}
    \subsubsection{Определение точки перегиба графика функции. Лемма о взаимном расположении графика функции и касательной к нему, проведенной в точке перегиба (с доказательством).}
    \subsubsection{Определение точки перегиба графика функции. Теорема: необходимое условие перегиба(с доказательством).}
    \subsubsection{Определение точки перегиба графика функции. Теорема: первое достаточное условие перегиба (с доказательством).}
    \subsubsection{Определение точки перегиба графика функции. Теорема: второе достаточное условие перегиба (с доказательством).}
    \subsubsection{Определение точки перегиба графика функции. Теорема: третье достаточное условие перегиба (с доказательством).}
    \subsubsection{Определение вертикальной и наклонной асимптот к графику функции. Теорема: правила нахождения наклонной асимптоты (с доказательством).}
    \subsection{Определенный интеграл}
    \subsubsection{Определение интегрируемости функции по Риману. Теорема: необходимое условие интегрируемости (с доказательством).}
    \subsubsection{Определение интегральной суммы, верхней и нижней сумм Дарбу. Леммы о соотношениях между ними (с доказательствами).}
    \subsubsection{Определение разбиения, диаметра разбиения. Лемма об изменении верхней суммы Дарбу при добавлении к разбиению новых точек (с доказательством).}
    \subsubsection{Определение верхнего и нижнего интегралов Дарбу. Леммы о соотношении между верхней и нижней суммами Дарбу, соответствующими различным разбиениям, и о соотношении между суммами Дарбу и интегралами Дарбу (с доказательствами).}
    \subsubsection{Основная лемма Дарбу (с доказательством).}
    \subsubsection{Критерий интегрируемости функции по Риману на отрезке (в терминах верхней и нижней сумм Дарбу, с доказательством).}
    \subsubsection{Теорема об интегрируемости непрерывной функции на отрезке (с доказательством).}
    \subsubsection{Теорема об интегрируемости на отрезке функции, имеющей разрывы (с доказательством).}
    \subsubsection{Теорема об интегрируемости функции, монотонной на отрезке (с доказательством).}
    \subsubsection{Теорема об интегрируемости композиции функций на отрезке (с доказательством).}
    \subsubsection{Свойства определенного интеграла: интегрируемость суммы, разности, произведения функции и константы, произведения функций (с доказательством).}
    \subsubsection{Свойства определенного интеграла: интегрируемость на подотрезке, аддитивность (с доказательством).}
    \subsubsection{Интегрирование неравенств, оценка модуля от интеграла (с доказательством).}
    \subsubsection{Первая теорема о среднем значении интеграла (с доказательством).}
    \subsubsection{Определение интеграла с переменным верхним пределом. Теорема: непрерывность интеграла с переменным верхним пределом (с доказательством).}
    \subsubsection{Определение интеграла с переменным верхним пределом. Теорема: дифференцируемость интеграла с переменным верхним пределом (с доказательством).}
    \subsubsection{Определение интеграла с переменным верхним пределом. Теорема: формула Ньютона-Лейбница (с доказательством).}
    \subsubsection{Формулы замены переменной и интегрирования по частям в определенном интеграле (с доказательством).}
    \subsection{Геометрические приложения определенного интеграла}
    \subsubsection{Определение длины плоской кривой. Теорема о спрямляемости и длине дуги кривой, заданной параметрически (с доказательством).}
    \subsubsection{Определение площади плоской фигуры. Два критерия квадрируемости (через приближение простейшими и квадрируемыми (с доказательством).}
    \subsubsection{Определение площади плоской фигуры. Теорема о квадрируемости и площади криволинейной трапеции (с доказательством).}
    \subsubsection{Определение площади плоской фигуры. Теорема о квадрируемости и площади криволинейного сектора (с доказательством).}
    \subsubsection{Определение объема тела в пространстве. Два критерия кубируемости (через приближение простейшими и кубируемыми, с доказательством).}
    \subsubsection{Определение объема тела в пространстве. Кубируемость цилиндрического тела и ступенчатого тела (с доказательством).}
    \subsubsection{Определение объема тела в пространстве. Кубируемость тела, образованного вращением криволинейной трапеции вокруг оси Ox (с доказательством).}
    \subsection{Несобственные интегралы}
    \subsubsection{Определение несобственного интеграла первого рода. Теоремы: замена переменной и интегрирование по частям в несобственном интеграле первого рода (с доказательствами).}
    \subsubsection{Критерий Коши и признак сравнения сходимости несобственного интеграла первого рода (с доказательствами).}
    \subsubsection{Определение абсолютной и условной сходимости несобственного интеграла первого рода. Признак Дирихле-Абеля сходимости несобственного интеграла первого рода (с доказательством).}
    \subsubsection{Определение несобственного интеграла первого и второго рода. Понятие главного значения несобственного интеграла первого и второго рода. Главное значение интеграла первого рода для нечетной функции.}
    \subsubsection{Метод хорд нахождения корня уравнения и его обоснование.}
    \subsubsection{Метод касательных нахождения корня уравнения и его обоснование.}
    \subsubsection{Метод прямоугольников вычисления определенного интеграла (с выводом оценки погрешности).}
    \subsubsection{Формула Тейлора с остаточным членом в интегральной форме (с доказательством).}
    \subsection{Функции многих переменных}
    \subsubsection{Определение предела последовательности в пространстве $\mathbb{R}_m$. Леммы об эквивалентности сходимости последовательности в пространстве $\mathbb{R}_m$ и покоординатной сходимости, фундаментальности в пространстве $\mathbb{R}_m$ и покоординатной фундаментальности (с доказательствами).}
    \subsubsection{Теорема Больцано-Вейерштрасса для последовательностей точек пространства $\mathbb{R}_m$ (с доказательством).}
    \subsubsection{Определение предела функций многих переменных по Гейне и по Коши. Доказательство их эквивалентности.}
    \subsubsection{Критерий Коши существования предела функции многих переменных в точке (с доказательством).}
    \subsubsection{Определение сложной функции многих переменных. Теорема о непрерывности сложной функции многих переменных (с доказательством).}
    \subsubsection{Теорема о прохождении непрерывной функции многих переменных через промежуточное значение (с доказательством).}
    \subsubsection{Первая и вторая теоремы Вейерштрасса для функции многих переменных (с доказательствами).}
    \subsubsection{Определение равномерной непрерывности функции многих переменных. Теорема Кантора для функции многих переменных (с доказательством).}
    \subsection{Дифференциальное исчисление ФМП}
    \subsubsection{Определение дифференцируемости функции многих переменных в точке. Необходимое условие дифференцируемости, связь между дифференцируемостью и непрерывностью (с доказательствами).}
    \subsubsection{Достаточное условие дифференцируемости функции многих переменных в точке (с доказательством).}
    \subsubsection{Определение касательной плоскости к поверхности в трехмерном пространстве. Вывод уравнения касательной плоскости.}
    \subsubsection{Теорема о дифференцируемости сложной функции (с доказательством).}
    \subsubsection{Теорема: инвариантность формы первого дифференциала (с доказательством).}
    \subsubsection{Определение производной по направлению и градиента функции многих переменных. Геометрический смысл градиента (с доказательством).}
    \subsubsection{Определение частной производной высокого порядка. Теорема о достаточных условиях независимости смешанных производных от порядка дифференцирования (теорема Юнга, с доказательством).}
    \subsubsection{Определение частной производной высокого порядка. Теорема о достаточных условиях независимости смешанных производных от порядка дифференцирования (теорема Шварца, с доказательством).}
    \subsubsection{Формула Тейлора с остаточным членом в форме Лагранжа (с доказательством).}
    \subsubsection{Формула Тейлора с остаточным членом в форме Пеано (формулировка и доказательство леммы 1).}
    \subsubsection{Формула Тейлора с остаточным членом в форме Пеано (формулировка и доказательство леммы 2).}
    \subsubsection{Определение локального экстремума функции многих переменных. Необходимое условие экстремума функции многих переменных. (с доказательством).}
    \subsubsection{Определение локального экстремума функции многих переменных. Доказательство того, что если второй дифференциал функции в точке является знакопеременной КФ, то экстремума в данной точке нет.}
    \subsection{Функции заданные неявно. Экстремум}
    \subsubsection{Формулировка теоремы о существовании, непрерывности и дифференцируемости неявной функции, заданной одним уравнением. Доказательство существования и единственности.}
    \subsubsection{Формулировка теоремы о существовании, непрерывности и дифференцируемости неявной функции, заданной одним уравнением. Доказательство дифференцируемости.}
    \subsubsection{Теорема о разрешимости системы функциональных уравнений (с доказательством)}
    \subsubsection{Определение зависимости функций. Достаточные условия независимости функций.}
    \subsubsection{Определение зависимости функций. Теорема о функциональных матрицах.}
    \subsubsection{Определение условного локального экстремума функции многих переменных. Вывод необходимых условий экстремума (без функции Лагранжа).}
    \subsubsection{Определение условного локального экстремума функции многих переменных. Вывод необходимых условий экстремума в терминах множителей Лагранжа.}
    \subsubsection{Определение условного локального экстремума функции многих переменных. Вывод достаточных условий экстремума в терминах множителей Лагранжа.}
\end{document}