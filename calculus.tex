\documentclass[10pt]{article}
\pdfoutput=1
\usepackage[T2A]{fontenc}
\usepackage[utf8]{inputenc}
\usepackage[english, russian]{babel}
\usepackage{NotesTeX_rus}

\begin{document}
    \tableofcontents
    \section{Определения}
    \begin{definition}
        Вектор $P = (x_0, \ldots, x_n)$ называется \textbf{разбиением(неразмеченным)} отрезка $[a, b]$, если $a = x_0 < x_1 < \ldots < x_n = b$, $i$-й отрезок разбиения обозначаем $\Delta_i := [x_{i-1}, x_i]$  
    \end{definition}
    \begin{definition}
        Разбиение $P'$ отрезка $[a, b]$ называется \textbf{измельчением} разбиения $P$ того же отрезка, если все элементы $P$ так же являются элементами $P'$
    \end{definition}
    \begin{definition}
        Разбиение $R$ называется \textbf{объединением разбиений} $P$ и $Q$ отрезка $[a, b]$, если $R$ включает в себя все элементы $P$ и $Q$ и только их. Обозначение: $R = P \cup Q$
    \end{definition}
    \begin{definition}
        Пусть функция $f$ определена на отрезке $[a, b]$, $\{x_k\}$ - его размеченное разбиение. Тогда число
        $$
        \sigma(f, P, \xi) = \sum_{i=1}^n f(\xi_i)\Delta x_i
        $$
        где $\Delta x_i = |\Delta_i| := x_i - x_{i-1}$, a $\xi \in \Delta x_i$, называется \textbf{интегральной суммой} функции $f$, соответствующей размеченному разбиению $\{x_k\}$
    \end{definition}
    \begin{definition}
        Пусть $P$ - разбиение отрезка $[a, b]$. Тогда числа
        \begin{gather*}
            S_{*}(P) = \underline{S}(P) := \sum_{i = 1}^{n(P)} m_i \Delta x_i \\
            S^{*}(P) = \overline{S}(P) := \sum_{i = 1}^{n(P)}M_i \Delta x_i\\
            \text{где }m_k := \inf_{x \in \Delta_k}f(x);\, M_k := \sup_{x \in \Delta_k}f(x)
        \end{gather*}
        называются \textbf{нижней} и \textbf{верхней суммами Дарбу} функции $F$ для разбиения $P$ 
    \end{definition}
    % TODO Определение предела интегральных сумм при стремлении диаметра разбиения к нулю на языке \varepsilon-\delta
    \section{Билеты}
    % -----------------------------------------------------------------------------------
    \subsection{Применение дифференциального исчисления для построения графиков}
    \subsubsection{Определение: строгого локального экстремума функции одной переменной. Теорема: первое достаточное условие экстремума (с доказательством).}
    \begin{definition}
        Пусть функция $f(x)$ определена всюду в некоторой окрестности точки $c$. Тогда эта функция имеет в точке с \textbf{локальный экстремум}, если существует такая окрестность точки $c$, что для всех точек этой окрестности значение $f(c)$ является наибольшим или наименьшим среди всех значений $f(x)$ этой функции
    \end{definition}
    \begin{theorem}[первое достаточное условие экстремума]
        Пусть $f(x)$ дифференцируема всюду в некоторой окрестности точки $c$, и пусть точка $c$ является стационарной точкой функции $f(x)$. Тогда если в пределах указанной окрестности производная $f(x)'$ положительна(отрицательна) слева от точки $c$ и отрицательна(положительна) справа от точки $c$, то функция $f(x)$ имеет в точке $c$ локальный максимум(минимум). Если же в пределах указанной окрестности точки $c$ производная $f(x)'$ имеет один и тот же знак слева и справа от точки $c$, то экстремума в точке $c$ нет
    \end{theorem}
    \begin{proof}
        \begin{enumerate}
            \item Пусть $f(x)'$ в пределах рассматриваемой окрестности положительна(отрицательна) слева и отрицательна(положительная) справа от $c$. Обозначим через $x_0$ любое значение аргумента из рассматриваемой окрестности, отличное от $c$. Достаточно доказать, что
            \begin{gather*}
                f(c)-f(x_0)>0\,(<0)
            \end{gather*}
            Т.к. $f(x)$ дифференцируема всюду на $U(c)$, то она дифференцируема на сегменте, ограниченном $c$ и $x_0 \Rightarrow$ выполнены все условия теоремы Лагранжа и
            \begin{gather}
                f(c) - f(x_0) = f'(\xi)(c-x_0)\label{eq:lagrange}
            \end{gather}
            где $\xi$ - некоторое значение аргумента между $c$ и $x_0$. Поскольку производная $f'(\xi)$ положительна(отрицательна) при $x_0 < c$ и отрицательна(положительна) при $x_0 > c$, правая часть \ref{eq:lagrange} положительна(отрицательна)
            \item Пусть теперь производная $f'(x)$ имеет один и тот же знак слева и справа от $c$. Обозначая через $x_0$ любое значение аргумента, отличное от $c$ и повторяя рассуждения выше, мы докажем, что правая часть \ref{eq:lagrange} имеет разные знаки при $x_0 < c(x_0 > c) \Rightarrow$ в $c$ нет экстремума.
        \end{enumerate}
    \end{proof}
    % -----------------------------------------------------------------------------------



    \subsubsection{Определение: строгого локального экстремума функции одной переменной. Теорема: второе достаточное условие экстремума (с доказательством).}
    \begin{definition}
        Пусть функция $f(x)$ определена всюду в некоторой окрестности точки $c$. Тогда эта функция имеет в точке с \textbf{локальный экстремум}, если существует такая окрестность точки $c$, что для всех точек этой окрестности значение $f(c)$ является наибольшим или наименьшим среди всех значений $f(x)$ этой функции
    \end{definition}
    \begin{theorem}[второе достаточное условие экстремума]
        Пусть $f(x)$ имеет в данной стационарной точке $c$ конечную вторую производную. Тогда функция $f(x)$ имеет в точке $c$ локальный максимум(минимум), если $f^{(2)} < 0\,(>0)$ 
    \end{theorem}
    \begin{proof}
        Из условия $f^{(2)} < 0\,(>0)$ и из доказанной в гл.6 теоремы 6.1 вытекает, что функция $f'(x)$ убывает(возрастает) в точке $c$. Поскольку по условию $f'(c) = 0$, то найдется такая окрестность точки $c$, в пределах которой $f'(x)$ положительна(отрицательна) слева от $c$ и отрицательна(положительна) справа от $c$. Но тогда по первому достаточну условию экстремума $f(x)$ имеет в точке $c$ локальный максимум(минимум)
        % TODO: настроить ссылку на теорему 6.1
    \end{proof}
    % -----------------------------------------------------------------------------------



    \subsubsection{Определение: строгого локального экстремума функции одной переменной. Теорема: третье достаточное условие экстремума (с доказательством).}
    \begin{definition}
        Пусть функция $f(x)$ определена всюду в некоторой окрестности точки $c$. Тогда эта функция имеет в точке с \textbf{локальный экстремум}, если существует такая окрестность точки $c$, что для всех точек этой окрестности значение $f(c)$ является наибольшим или наименьшим среди всех значений $f(x)$ этой функции
    \end{definition}
    \begin{theorem}[третье достаточное условие экстремума]
        Пусть $n \geq 1$ - некоторое неяетное число , и пусть функция $y = f(x)$ имеет производную порядка $n$ в некоторой окрестности точки $c$ и производную порядка $n+1$ в самой точке $c$. Тогда, если
        \begin{gather}
            f'(c) = f^{(2)}(c) = \ldots = f^{(n)}(c) = 0,\, f^{(n+1)} \neq 0 \label{eq:extr_cond}
        \end{gather}
        то $y = f(x)$ имеет в точке $c$ локальный максимум(минимум) при $f^{(n+1)} < 0\,(>0)$
    \end{theorem}
    \begin{proof}
        При $n = 1$ теорема совпадает со вторым достаточным условием экстремума, так что достаточно доказать для нечетного $n \geq 3$ \\
        Пусть $n \geq 3;\; f^{(n+1)} > 0$. Докажем, что $y = f(x)$ имеет в $c$ локальный минимум. Доказательства для остальных случаев проводятся аналогично\\
        Так как $f^{(n+1)} > 0$, то, в силу теоремы 6.1 о достаточном условии возрастании функции в точке, функция $f^{(n)}(x)$ отрицательная слева от $c$ и положительна справа от $c$. Тогда разложим $f(x)$ в окрестности $c$ с остаточным членом в форме Лагранжа. Тогда для всех $x$ из достаточно малой окрестности $c$ между $x$ и $c$ найдется $\xi$ такая, что
        \begin{gather*}
            f'(x) = f'(c)+\frac{f^{(2)}(c)}{1!}(x-c) + \ldots + \frac{f^{(n-1)}(c)}{(n-2)!}(x-c)^{n-2} + \frac{f^{(n)}(\xi)}{(n-1)!}(x-c)^{n-1}
        \end{gather*}
        В силу соотношений \ref{eq:extr_cond} написанное разложение принимает вид
        \begin{gather}
            f'(x) = \frac{f^{(n)}(\xi)}{(n-1)!} (x-c)^{n-1} \label{eq:taylor_simp}
        \end{gather}
        Так как $\xi$ лежит между $x$ и $c$, то для всех $x$ из достаточно малой окрестности точки $c$ велчичина $f^{(n)}(\xi)$(а значит, так как $n$ нечетная - и вся правая часть \ref{eq:taylor_simp}) отрицательная слева от $c$ и положительна справа от $c$\\
        В силу первого достаточного условия экстремума $y = f(x)$ имеет в $c$ локальный минимум 
        % TODO: ссылка на теорему 6.1
    \end{proof}
    % -----------------------------------------------------------------------------------



    \subsubsection{Определение: график функции на данном интервале имеет выпуклость, направленную вверх (вниз). Теорема о связи между знаком второй производной и выпуклостью (с доказательством).}
    \begin{definition}
        График функции $y = f(x)$ имеет на интервале $(a, b)$ \textbf{выпуклость, направленную вверх(вниз)}, если график этой функции в пределах указанного интервала лежит не выше(не ниже) любой своей касательной
    \end{definition}
    \begin{remark}
        Термин "график лежит не выше(не ниже) своей касательной" имеет смысл, т.к. касательная не параллельна $Ox$
    \end{remark}
    \begin{theorem}[о связи между знаком второй производной и выпуклостью]
        Если функция $y = f(x)$ имеет на интервале $(a, b)$ конечную вторую производную и если эта производная неотрицательна(неположительна) всюду на этом интервале, то график функции $y = f(x)$ имеет на интервале $(a, b)$ выпуклость, направленную вниз(вверх)
    \end{theorem}
    \begin{proof}
        Ради определнности рассмотрим случай, когда вторая производная $f^{(2)} \geq 0$ всюду на $(a, b)$. Обозначим через $c$ любую точку интервала $(a, b)$. Требуется доказать, что график функции $y = f(x)$ в пределах интервала $(a, b)$ лежит не ниже касательной, проходящей через точку $M(c, f(c))$. Запишем уравнение указанной касательной, обоззначая ее текущую ординату через $Y$. Поскольку угловой коэфициент указанной касательной равен $f'(c)$, то ее уравнение имеет вид
        \begin{gather}
            Y - f(c) = f'(c)(x-c) \label{eq:angle_coef}
        \end{gather}
        Разложим функцию $f(x)$ в окрестности точки $c$ по формули Тейлора, беря в этой формуле $n-1$. Получим
        \begin{gather}
            y = f(x) = f(c) + \frac{f'(c)}{1!}(x-c) + \frac{f^{(2)}(\xi)}{2!}(x-c)^2 \label{eq:taylor_n-1}
        \end{gather}
        где остаточный член взят в форме Лагранжа, $\xi$ заключена между $c$ и $x$\\
        Сопостовляя \ref{eq:taylor_n-1} и \ref{eq:angle_coef} будем иметь
        \begin{gather}
            y-Y=\frac{f^{(2)}(\xi)}{2}(x-c)^2 \label{eq:convex_eq}
        \end{gather}
        Поскольку вторая производная по условию неотрицательна всюду на $(a, b)$, то правая часть \ref{eq:convex_eq} неотрицательна, т.е. для всех $x$ из $(a, b)$ лежит не ниже касательной \ref{eq:angle_coef}\\
        Аналогично для $f^{(2)}(x) \leq 0$
    \end{proof}
    % -----------------------------------------------------------------------------------



    \subsubsection{Определение точки перегиба графика функции. Лемма о взаимном расположении графика функции и касательной к нему, проведенной в точке перегиба (с доказательством).}
    \begin{definition}
        Точка $M(c, f(c))$ графика функции $y = f(x)$ называется \textbf{точкой перегиба} этого графика, если существует такая окрестность точки $c$ оси абсцисс, в пределах которой график функции $y = f(x)$ слева и справа от $c$ имеет разные направления выпуклости.
    \end{definition}
    \begin{lemma}
        Пусть функция $y = f(x)$ имеет производную $f'(x)$ всюду в $\delta$-окрестности точки $c$. Тогда, если график функции $y = f(x)$ имеет на интервале $(c, c+\delta)$ выпуклость, направленную вниз(вверх), то всюду в пределах интервала $(c, x+\delta)$ этот график лежит не ниже(не выше) касательной к графику, проведенной в точке $M(c, f(c))$
    \end{lemma}
    \begin{proof}
        Рассмотрим последовательность $\{x_n\}$ точек интервала $(c, c+\delta)$, сходящуюся к точке $c$. Через каждую точку $M_n(x_n, f(x_n))$ графика функции $y = f(x)$ проведем касательную к этому графику, т.е. прямую
        \begin{gather*}
            Y_n = f(x_n) + f'(x_n)(x - x_n)
        \end{gather*}
        Так как по условию график функции $y = f(x)$ имеет на интервале $(c, c+\delta)$ выпуклость, направленную вниз(вверх), то для любого $n$ и любой фиксированной точки $x$ интервала $(c, c+\delta)$
        \begin{gather}
            f(x) - Y_n=f(x)-f(x_n) - f'(x_n)(x-x_n) \geq 0\; (\leq 0) \label{ineq: convex}
        \end{gather}
        Из условия непрерывности $f'(x)$(и тем более $f(x)$) в точке $c$ и из определения непрерывности по Гейне вытекает, что существует предел
        \begin{gather*}
            \lim_{n \to \infty}(f(x) - Y_n) = \lim_{n \to \infty}\{f(x) - f(x_n) - f'(x_n)(x-x_n)\} = f(x) - f(c) - f'(c)(x-c)
        \end{gather*}
        Из существования последнего предела в силу неравенства \ref{ineq: convex} и теоремы 3.13 мы получим, что
        \begin{gather*}
            f(x)-f(c)-f'(c)(x-c) \geq 0\; (\leq 0)
        \end{gather*}
        Если обозначить через $Y$ текущую ординату касательной \ref{eq:angle_coef}, проходящей через точку $M(c, f(c))$, то последнее неравенство можно переписать в виде
        \begin{gather*}
            f(x) - Y \geq 0\; (\leq 0)
        \end{gather*}
        Итак, переходя \ref{ineq: convex} к пределу при $n \to \infty$ и используя теорему 3.13, мы получим, что $f(x) - Y \geq 0\, (\leq 0)$ для любой фиксированной точки $x$ из интервала $(c, c+\delta)$, причем $Y$ обозначает текущую ординату касатеьной, проведенной через точку $M(c, f(c))$
        % TODO: настроить ссылку на 3.13
        \end{proof}
        \begin{remark}
            Аналогично формулируется и доказывается лемма 1 и для случая когда график функции имеет определенное направление выпоклости на интервале $(c, c+\delta)$, а на интервале $(c-\delta, c)$
        \end{remark}
        \begin{lemma}
            % TODO: Лемма 2?
        \end{lemma}
    % -----------------------------------------------------------------------------------



    \subsubsection{Определение точки перегиба графика функции. Теорема: необходимое условие перегиба(с доказательством).}
    \begin{definition}
        Точка $M(c, f(c))$ графика функции $y = f(x)$ называется \textbf{точкой перегиба} этого графика, если существует такая окрестность точки $c$ оси абсцисс, в пределах которой график функции $y = f(x)$ слева и справа от $c$ имеет разные направления выпуклости.
    \end{definition}
    \begin{theorem}[необходимое условие перегиба(дважды дифференцируемой функции)]
        Если функция $y = f(x)$ имеет в точке $c$ вторую производную и график этой функции имеет перегиб в точке $M(c, f(c))$, то $f^{(2)}(c) = 0$
    \end{theorem}
    \begin{proof}
        Пусть $Y$ - текущая ордината касательной $Y=f(c)+f'(c)(x-c)$, проходящей через точку графика $M(c, f(c))$\\
        Рассмотрим функцию:
        \begin{gather*}
            F(x)=f(x) - Y = f(x) - f(c) - f'(c)(x-c)
        \end{gather*}
        равную разности $f(x)$ и линейной функции $f(c)+f'(c)(x-c)$\\
        $F(x)$, как и $f(x)$, имеет в точке $c$ вторую производную. В силу леммы 2 в малой окрестности точки $c$ график функции $y = f(x)$ лежит слева и справа от $c$ по разные стороны от касательной, проходящей через точку $M(c, f(c))$, а потому функция $F(x)$ в малой окрестности точки $c$ имеет слева и справа от $c$ разные знаки.\\
        Значит, $F(x)$ не может иметь в точке $c$ локального экстремума\\
        Предположим, что $f^{(2)} \neq 0$. Тогда, поскольку $F'(x) = f'(x) - f'(c),\, F^{(2)}(x) = f^{(2)}(2)$, выполняются условия $F'(c) = 0,\,F^{(2)}(c \neq 0)$ и функция $F(x)$ в силу второго достаточного условия экстремумаы имеет в точке $c$ локальный экстремум. Получаем противоречие $Rightarrow f^{(2)}(c) = 0$ 
    \end{proof}
    % -----------------------------------------------------------------------------------



    \subsubsection{Определение точки перегиба графика функции. Теорема: первое достаточное условие перегиба (с доказательством).}
    \begin{definition}
        Точка $M(c, f(c))$ графика функции $y = f(x)$ называется \textbf{точкой перегиба} этого графика, если существует такая окрестность точки $c$ оси абсцисс, в пределах которой график функции $y = f(x)$ слева и справа от $c$ имеет разные направления выпуклости.
    \end{definition}
    \begin{theorem}[первое достаточное условие перегиба]
        Пусть функция $y = f(x)$ имеет вторую производную в некоторой окрестности точки $c$ и $f^{(2)}(c) = 0$. Тогда, если в пределах указанной окрестности вторая производная $f^{(2)}(x)$ имеет разные знаки слева и справа от $c$, то график этой функции имеет перегиб в точке $M(c, f(c))$
    \end{theorem}
    \begin{proof}
        Заметим, что $y = f(x)$ имеет касательную в точке $M(c, f(c))$, ибо в $f'(c)$ сущестует конечная производная. Так как $f^{(2)}(x)$ слева и справа имеет разные знаки, из теоремы 7.5 заключаем, что направление выпуклости слева и справа от $c$ является различным.
        % TODO: настроит ссылку на 7.5
    \end{proof}
    % -----------------------------------------------------------------------------------



    \subsubsection{Определение точки перегиба графика функции. Теорема: второе достаточное условие перегиба (с доказательством).}
    \begin{definition}
        Точка $M(c, f(c))$ графика функции $y = f(x)$ называется \textbf{точкой перегиба} этого графика, если существует такая окрестность точки $c$ оси абсцисс, в пределах которой график функции $y = f(x)$ слева и справа от $c$ имеет разные направления выпуклости.
    \end{definition}
    \begin{theorem}[второе достаточное условие перегиба]
        Если функция $y = f(x)$ имеет в точке конечную третью производную и удовлетворяет в этой точке условиям $f^{(2)}(c) = 0;\,f^{(3)}(c) \neq 0$, то график этой функции имеет перегиб в точке $M(c, f(c))$
    \end{theorem}
    \begin{proof}
        Так как $f^{(3)}(c) \neq 0$ и из теоремы 6.1 вытекает, что из функция $f^{(2)}(x)$ либо возрастает, дибо убывает в точке $c$. Так как $f^{(2)}(c) = 0$, то и в том, и в другом случае найдется такая окрестность точки $c$, в пределах которой $f^{(2)}(x)$ имеет разные знаки слева и справа от $c$. Но тогда по предыдущей теореме график функции $y = f(x)$ имеет перегиб в точке $M(c, f(c))$
    \end{proof}
    % -----------------------------------------------------------------------------------



    \subsubsection{Определение точки перегиба графика функции. Теорема: третье достаточное условие перегиба (с доказательством).}
    \begin{definition}
        Точка $M(c, f(c))$ графика функции $y = f(x)$ называется \textbf{точкой перегиба} этого графика, если существует такая окрестность точки $c$ оси абсцисс, в пределах которой график функции $y = f(x)$ слева и справа от $c$ имеет разные направления выпуклости.
    \end{definition}
    \begin{theorem}[третье достаточное условие перегиба]
        Пусть $n \geq 2$ некоторое четное число, и пусть функция $y = f(x)$ имеет производную порядка $n$ в некоторой окрестности точки $c$ и производную порядка $n+1$ в самой точке $c$. Тогда, если выполнены соотношения
        \begin{gather}
            f^{(2)}(c)=f^{(3)}(c)=\ldots=f^{(n)}(c)=0, \quad f^{(n+1)}(c) \neq 0 \label{eq:conv_cond}
        \end{gather}
        то график функции $y = f(x)$ имеет перегиб в точке $M(c, f(c))$
    \end{theorem}
    \begin{proof}
        При $n=2$ теорема совпадает со вторым достаточным условием перегиба, так что нужно вести доказательство лишь для четного $n \geq 4$\\
        Пусть четное число $n$ удовлетворяет условию $n \geq 4$, и пусть $f^{(n+1)}(c) \neq 0$. Тогда, в силу теоремы 6.1 о достаточном условии возрастания или убывания функции в точке, функция $f^{(n)}(x)$ либо убывет в точке $c$ (при $f^{(n+1)}(c) < 0$), либо возрастает в этой точке (при $f^{(n+1)}(c) > 0$). Поскольку, кроме того, $f^{(n)}(c) = 0$, то и в том, и в другом случае всюду в достаточно малой окрестности точки $c$ функция $f^{(n)}(x)$ имеет разные знаки справа и слева от $c$\\
        Заметив это, разложим функцию $f^{(2)}(x)$ в окрестности точки $c$ по формуле Тейлора, записав остаточный член в форме Лагранжа. Мы получим, что для всех $x$ из достаточно малой окрестности точки $c$ между $x$ и $c$ найдется точка $\xi$ такая, что
        \begin{gather*}
            f^{(2)}(x) = f^{(2)}(c) + \frac{f^{(3)}(c)}{1!}(x-c) + \ldots + \frac{f^{(n-1)}(c)}{(n-3)!}(x-c)^{n-3} + \frac{f^{(n)}(\xi)}{(n-2)!}(x-c)^{n-2}
        \end{gather*}
        В силу соотношений \ref{eq:conv_cond} написанное разложение принимает вид
        \begin{gather}
            f^{(2)}(x) = \frac{f^{(n)}(\xi)}{(n-2)!}(x-c)^{n-2} \label{eq:taylor_simp2}
        \end{gather}
        Выше мы установили, что для всех $x$ из достаточно малой окрестности точки $c$ производная $f^{(n)}(x)$ имеет разные знаки справа и слева от $c$. Так как $\xi$ лежит между $x$ и $c$ то для всех x из достаточно малой окрестности точки $c$ величина $f^{(n)}(\xi)$(а значит, и вся правая часть \ref{eq:taylor_simp2}) имеет разные знаки слева и справа от $c$. В силу теоремы 7.8 график функции $y = f(x)$ имеет перегиб в точке $M(c, f(c))$
        % TODO: настроить ссылку на 7.8
    \end{proof}
    % -----------------------------------------------------------------------------------



    \subsubsection{Определение вертикальной и наклонной асимптот к графику функции. Теорема: правила нахождения наклонной асимптоты (с доказательством).}
    \begin{definition}
        Говорят, что прямая $x = a$ является \textbf{вертикальной асимптотой} графика функции $y = f(x)$, если хотя бы один из пределов
        \begin{gather*}
            \lim_{x \to a+0}f(x) \text{ или } \lim_{x \to a-0}f(x)
        \end{gather*}
        равен $+\infty$ или $-\infty$
    \end{definition}
    \begin{definition}
        Говорят, что прямая
        \begin{gather*}
            Y=kx+b
        \end{gather*}
        является \textbf{наклонной асимптотой} графика функции $y = f(x)$ при $x \to \infty$, если $f(x)$ пердставима в виде
        \begin{gather*}
            f(x) = kx+b+\alpha(x)
        \end{gather*}
        где
        \begin{gather*}
            \lim_{x \to +\infty} \alpha(x) = 0
        \end{gather*}
    \end{definition}
    \begin{theorem}
        Для того чтобы график функции $y = f(x)$ имел $x \to \infty$ наклонную асимптоту $Y=kx+b$, необходимо и достаточно, чтобы существовали два предела
        \begin{gather*}
            \lim_{x \to \infty} \frac{f(x)}{x} = k \text{ и } \lim_{x \to \infty} (f(x) - kx) = b
        \end{gather*}
    \end{theorem}
    \begin{proof}
        $\Rightarrow$
        Пусть график функции $y = f(x)$ имеет при $x \to +\infty$ асимптоту $Y=kx+b$, т.е. для $f(x)$ справедливо представление $f(x) = kx+b+\alpha(x)$. Тогда
        \begin{gather*}
        \lim_{x \to +\infty} \frac{f(x)}{x} = \lim_{x \to +\infty} \frac{kx + b + \alpha(x)}{x} = \lim_{x \to +|infty}(k + \frac{b}{x} + \frac{\alpha(x)}{x}) = k \\
        \lim_{x \to \infty}(f(x) - kx) = \lim_{x \to \infty}(b+\alpha(x)) = b
        \end{gather*}
        $\Leftarrow$
        Пусть существуют пределы в условии. Второй из этих пределов дает право утверждать, что $f(x) - kx - b$ является бесконечно малой при $x \to \infty$. Обозначив эту б.м. через $\alpha(x)$, получим для $f(x)$ представление $f(x) + b + \alpha(x)$
        \begin{remark}
            Аналогично доказывается и для наклонной асимптоты и для случая $x \to -\infty$
        \end{remark}
    \end{proof}
    % -----------------------------------------------------------------------------------



    \subsection{Определенный интеграл}
    \subsubsection{Определение интегрируемости функции по Риману. Теорема: необходимое условие интегрируемости (с доказательством).}
    \begin{definition}
        Функция $f(X)$ называется \textbf{интегрируемой по Риману на сегменте $[a, b]$}, если для этой функции на указанном сегменте существует предел $I$ ее интегральных сумм $\sigma$ при стремлении диаметра $d$ разбиений $\{x_k\}$ к нулю. Обозначение: $\int_a^b f(x)dx$
    \end{definition}
    % TODO: разобраться что за условие
    % -----------------------------------------------------------------------------------



    \subsubsection{Определение интегральной суммы, верхней и нижней сумм Дарбу. Леммы о соотношениях между ними (с доказательствами).}
    \begin{definition}
        Пусть функция $f$ определена на отрезке $[a, b]$, $\{x_k\}$ - его размеченное разбиение. Тогда число
        $$
        \sigma(f, P, \xi) = \sum_{i=1}^n f(\xi_i)\Delta x_i
        $$
        где $\Delta x_i = |\Delta_i| := x_i - x_{i-1}$, a $\xi \in \Delta x_i$, называется \textbf{интегральной суммой} функции $f$, соответствующей размеченному разбиению $\{x_k\}$
    \end{definition}
    \begin{definition}
        Пусть $P$ - разбиение отрезка $[a, b]$. Тогда числа
        \begin{gather*}
            S_{*}(P) = \underline{S}(P) := \sum_{i = 1}^{n(P)} m_i \Delta x_i \\
            S^{*}(P) = \overline{S}(P) := \sum_{i = 1}^{n(P)}M_i \Delta x_i\\
            \text{где }m_k := \inf_{x \in \Delta_k}f(x);\, M_k := \sup_{x \in \Delta_k}f(x)
        \end{gather*}
        называются \textbf{нижней} и \textbf{верхней суммами Дарбу} функции $F$ для разбиения $P$ 
    \end{definition}
    \begin{lemma}
        Пусть $\sigma(x_n, \xi_k)$ - интегральная сумма, отвечающая данному разбиению $\{x_k\}$. Тогда при любом выборе промежуточных точек $\xi_k$ всегда справедливы неравенства
        \begin{gather*}
            s \leq \sigma \leq S
        \end{gather*}
        где $s$ и $S$ - соответственно нижняя и верхняя суммы, отвечающие тому же разбиению
    \end{lemma}
    \begin{proof}
        По определению чисел $m_k$ и $M_k$ заключаем, что $m_k \leq f(\xi_k) \leq M$ для любого $\xi_k$ из сегмента $[x_{k-1}, x_k]$. Умножая написанные неравенства на $\Delta x_k$ и суммируя по всем $k$ от $1$ до $n$, получаем требуемое утверждение леммы
    \end{proof}
    \begin{lemma}
        Пусть $\{x_n\}$ - произвольное фиксированное разбиение сегмента $[a, b], \varepsilon$ - произвольное положительное число. Тогда можно выбрать промежуточные точки $\xi_k$ так, чтобы интегральная сумма $\sigma(x_k, \xi_k)$ и верхняя сумма $S$ удовлетворяли неравенству $0 \leq S-\sigma(x_k, \xi_k) < \varepsilon$. Промежуточные точки $\mu_k$ vожно выбрать и таким образом, чтобы для интегральной суммы $\sigma(x_k, \mu_k)$ и нижней суммы $s$ выполнялись неравенства $0 \leq \sigma(x_k, \mu_k) - s < \varepsilon$
    \end{lemma}
    \begin{proof}
        Пусть $\{x_n\}$ - фиксированное разбиение сегмента $[a, b]$ и $\varepsilon > 0$. Докажем сначала первое утверждение леммы. Поскольку $M_k = \sup_{x_k-1 \leq x \leq x_k} f(x)$, то для выбранного нами $\varepsilon>0$ найдется точка $\xi_k$ сегмента $[x_{k-1}, x_k]$ такая, что $0 \leq M_k - f(\xi_k) < \frac{\varepsilon}{(b-a)}$. Умножив эти неравенства на $\Delta x_k$ и просуммировав по всем $k$ от $1$ до $n$, получим
        \begin{gather*}
            0 \leq S - \sigma(x_k, \xi_k) < \varepsilon
        \end{gather*}
        Аналогично в силу того, что $m_k = \inf_{x_k-1 \leq x \leq x_k} f(x)$, существует такая точка $\mu_k \in [x_{k-1}, x_k]$, что
        \begin{gather*}
            0 \leq f(\mu_k) - m_k < \frac{\varepsilon}{(b-a)}
        \end{gather*}
        Последние неравенства после умножения на $\Delta x_n$ и суммированния приводят к оценкам
        \begin{gather*}
            0 \leq \sigma(x_k, \mu_k) - s < \varepsilon
        \end{gather*}
    \end{proof}
    \begin{corollary}
        Для любого фиксированного разбиения $\{x_n\}$ справедливы следующие соотношения
        \begin{gather*}
            S = \sup_{\xi_k} \sigma(x_k, \xi_k),\, s = \inf_{\mu_k} \sigma(x_k, \mu_k)
        \end{gather*}
        где точная верхняя и нижняя грани берутся по всевозсожным промежуточным точкам
    \end{corollary}
    % -----------------------------------------------------------------------------------



    \subsubsection{Определение разбиения, диаметра разбиения. Лемма об изменении верхней суммы Дарбу при добавлении к разбиению новых точек (с доказательством).}
    \begin{definition}
        Вектор $P = (x_0, \ldots, x_n)$ называется \textbf{разбиением(неразмеченным)} отрезка $[a, b]$, если $a = x_0 < x_1 < \ldots < x_n = b$, $i$-й отрезок разбиения обозначаем $\Delta_i := [x_{i-1}, x_i]$  
    \end{definition}
    % TODO: определение диаметра разбиения
    \begin{lemma}
        При измельчении данного разбиения верхняя сумма может только уменьшиться, а нижняя - только увеличится
    \end{lemma}
    \begin{proof}
        Пусть $\{x_n\}$ - данное разбиение, а разбиение $\{x_n'\}$ получается из него добавлением только одной новой точки $\overline{x}$ лежит внутри $[x_{k-1}, x_k]$. Тогда в выражении для $S$ слагаемое $M_k \Delta x_k$ заменится на $M'_k(\overline{x}-x_{k-1}) + M''_k (x_k - \overline{x})$, где $M'_k = \sup_{x_{k-1} \leq x \leq \overline{x}} f(x),\,M''_k=\sup_{\overline{x} \leq x \leq x_k} f(x)$\\
        Точная верхняя грань функции на части сегмента не превосходит точной верхней грани функции на всем сегменте. Поэтому $M'_k \leq M_k,\,M''_k \leq M_k$ и
        \begin{gather*}
            M'_k(\overline{x} - x_{k-1}) + M''_k(x_k - \overline{x}) \leq M_k((\overline{x} - x_{k-1}) + (x_k -     \overline{x})) = M_k \Delta x_k
        \end{gather*}
        Так как все другие слагаемые в выражении для верхней суммы сохранятся, то мы дказали, что при добавлении точки $\overline{x}$ верхняя сумма может только уменьшиться. Случай, когда к данному разбиению добавляется несколько новых точек, сводится, очевидно, к рассмотренному. аналогично для нижней суммы. 
    \end{proof}
    % -----------------------------------------------------------------------------------



    \subsubsection{Определение верхнего и нижнего интегралов Дарбу. Леммы о соотношении между верхней и нижней суммами Дарбу, соответствующими различным разбиениям, и о соотношении между суммами Дарбу и интегралами Дарбу (с доказательствами).}
    \begin{definition}
        \textbf{Верхним интегралом Дарбу} от функции $f(x)$ называется число $I^*$, равное точной нижней грани множества верхних сумм $\{S\}$ данной функции $f(x)$ для всевозможных разбиений сегмента $[a, b]$\\
        \textbf{Нижним интегралом Дарбу} от функции $f(x)$ называется число $I_*$, равное точной нижней грани множества нижних сумм $\{s\}$ данной функции $f(x)$ для всевозможных разбиений сегмента $[a, b]$\\
    \end{definition}
    \begin{lemma}
        Для двух произвольных и, вообще говоря, различных разбиений сегмента $[a, b]$ нижняя сумма одного из этих разбиений не превосходит верхней суммы другого
    \end{lemma}
    \begin{proof}
        Пусть $\{x_k'\}$ и $\{x_k''\}$ - два произвольных разбиения сегмента $[a, b]$, а $S',\,s',\,S'',\,s''$ - верхние и нижние суммы этих разбиений соответственно. Обозначим через $\{x_k\}$ объединение разбиений $\{x_k'\}$ и $\{x_k''\}$, а через $S$ и $s$ верхнюю и нижнюю суммы разбиения $\{x_k\}$. Заметим, что $\{x_k\}$ является измельчением как разбиения $\{x_k'\}$ и $\{x_k''\}$. Согласно утверждению леммы 3 справдливы неравенства
        \begin{gather*}
            \underline{S' \geq  S},\, S'' \geq S,\, s' \leq s,\, \underline{s'' \leq s}
        \end{gather*}
        Кроме того, в силу леммы 1 получим, что $\underline{s \leq S}$. Пользуясь свойством транзитивности для числовых неравенств и используя три подчеркнутых выше неравенства, заключаем, что $s'' \leq S'$. Аналогично $s' \leq S''$
        % TODO: настроить ссылку на лемму 1 и лемму 3
    \end{proof}
    \begin{lemma}
        Нижний интеграл Дарбу всегда не превосходит верхнего интеграла Дарбу, т.е. $I_* \leq I^*$
    \end{lemma}
    % TODO: разобраться что тут
    % -----------------------------------------------------------------------------------


    
    \subsubsection{Основная лемма Дарбу (с доказательством).}
    \begin{lemma}
        Верхний интеграл Дарбу $I^*$ является пределом верхних сумм $S$ при стремлении диаметра $d$ разбиений к нулю, т.е. $I^*=\lim_{d \to 0} S$. Аналогично $I_*=\lim_{d \to 0} s$
    \end{lemma}
    \begin{proof}
        Проведем доказательство первого утверждения леммы. Заметим, что если функция $f(x) = const$, то $S=c(b-a)=I^*$ для любого разбиения. Поэтому $\lim_{d \to 0} S = I^*$. Если функция $f(x)$ непотсоянна, то $M=\sup_{x \in [a, b]}f(x) > m = \inf_{x \in [a, b]}f(x)$. Фиксируем произвольное положительное число $\varepsilon$. По определению числа $I^*$ существует такое разбиение $\{x_k^*\}$, что верхняя сумма $S^*$ этого разбиения будет удовлетворять условию $S^* - I^* < \frac{\varepsilon}{2}$. Обозначим через $l$ число точек разбиения $\{x_k^*\}$, не совпадающих с концами отрека $[a, b]$\\
        Пусть $\{x_k\}$ - произвольное разбиение сегмента $[a, b]$, диаметр которого удовлетворяет неравенству $d < \sigma = \frac{\varepsilon}{2l(M-m)}$, и пусть $S$ - верхняя сумма этого разбиения $\{x_k\}$, добавив к нему отмеченные выше $l$ точек разбиения $\{x_k*\}$. Полученное при этом разбиение обозначим символом $\{x_k'\}$. По лемме $6$ верхняя сумма $S'$ этого последнего разбиения удовлетворяет условию
        \begin{gather*}
            0 \leq S - S' \leq (M-m)ld < \frac{\varepsilon}{2}
        \end{gather*}
        Но разбиение $\{x_k'\}$ можно рассмотривать как измельчение разбиение $\{x_k*\}$, к которому добавляются точки разбиения $\{x_k\}$, не совпадающие с концами сегмента $[a, b]$. Поэтому в силу определения $I^*$ и леммы 3
        \begin{gather*}
            I^* \leq S' \leq S^* \text{, т.е. } 0 \leq S' - I^* < S^* - I^*
        \end{gather*}
        Выше было показано, что $S^* - I^* < \frac{\varepsilon}{2}$, поэтому $0 \leq S' - I^* < \frac{\varepsilon}{2}$. Объединяя эти неравенства с установленными выше неравенствами $0 \leq S - S' < \frac{\varepsilon}{2}$, получаем, что $0 \leq S - I^* < \varepsilon$, если только $d$ меньше указанного выше $\delta$. Следовательно, $I^* = \lim_{d \to S} S$. Для нижних сумм доказательство аналогично.
        % TODO:исправить лемму 3/лемму 6
    \end{proof}
    % -----------------------------------------------------------------------------------



    \subsubsection{Критерий интегрируемости функции по Риману на отрезке (в терминах верхней и нижней сумм Дарбу, с доказательством).}
    \begin{theorem}[вспомогательная теорема]
        Для того, чтобы ограниченная на сегменте $[a, b]$ функция $f(x)$ была интегрируемая на этом сегменте, необходимо и достаточно, чтобы выполнялось равенство $I_* = I^*$
    \end{theorem}
    \begin{proof}
        $\Rightarrow$
        Пусть функция $f(x)$ интегрируема по Риману на сегменте $[a, b]$. Тогда существует предел $I$ ее интегральных сумм $\sigma$ при стремлении диаметра $d$ разбиений к нулю \\
        По определению предела интегральных сумм для любого $\varepsilon > 0$ существует такое $\delta > 0$, что для любого выбора промежуточных точек $\xi_k$ разбиения $\{x_k\}$ с диаметром $d < \delta$ выполняется неравенство
        \begin{gather*}
            |I - \sigma(x_k, \xi_k)| < \frac{\varepsilon}{4}
        \end{gather*}
        Согласно лемме 2 для данного разбиения $\{x_k\}$ можно так выбрать промежуточные точки $\xi_k'$ и $\xi_k''$ в каждом частичном сегменте $[x_{k-1}, x_k]$, что будут справедливы неравенства
        \begin{gather*}
            S - \sigma(x_k, \xi_k') < \frac{\varepsilon}{4},\, \sigma(x_k, \xi_k'') - s < \frac{\varepsilon}{4}
        \end{gather*}
        Подчеркнем, что, кроме того, для данного разбиения $\{x_k\}$ одновременно выполнены неравенства
        \begin{gather*}
            |I - \sigma(x_k, \xi_k')| < \frac{\varepsilon}{4},\,|I - \sigma(x_k, \xi_k'')| < \frac{\varepsilon}{4}
        \end{gather*}
        Заметим теперь, что
        \begin{gather*}
            S-s=[S-\sigma(x_k, \xi_k')] + [\sigma(x_k, \xi_k') - I] + [I-\sigma(x_k, \xi_k'')] + [\sigma(x_k, \xi_k'' - s)]
        \end{gather*}
        Отсюда, учитывая, что модуль суммы четырех величин не превосходит суммы их модулей, получаем, что $S-s<\varepsilon$. Итак, для любого $\varepsilon > 0$ существует такое $\delta > 0$, что для любого разбиения с диаметром $d < \delta$ справеливо неравенство $S-s<\varepsilon$. Поскольку для любого разбиения выполнены неравенства
        \begin{gather*}
            s \leq I_* \leq I^* \leq S
        \end{gather*}
        то из неравенства $S-s < \varepsilon$ вытекает, что $0 \leq I^* - I_* < \varepsilon$, а отсюда в силу произвольности $\varepsilon > 0$ вытекает, что $I^* = I_*$.\\
        $\Leftarrow$
        Пусть $I^*=I_*=A$. Согласно основной лемме Дарбу $I^* = \lim_{d \to 0} S,\, I_*=\lim_{d \to 0} s$, т.е. верхний интеграл является пределом верхних сумм, а нижний интеграл - пределом нижних сумм при стремлении диаметра $d$ разбиений к нулю. Поэтому для любого $\varepsilon > 0$ можно указать такое число $\delta > 0$, что для любого разбиения с диаметром $d < \delta$ одновременно выполняются неравенства $I_* - s = A-s<\varepsilon,\;S-I^*=S-A<\varepsilon$. При любом указанном разбиении любая интегральная сумма $\sigma(x_k, \xi_k)$ удовлетворяет неравенствам $s \leq \sigma(x_k, \xi_k) \leq S$, а, значит, и неравенствам
        \begin{gather*}
            A - \varepsilon < s \leq \sigma(x_k, \xi_k) \leq S < A + \varepsilon
        \end{gather*}
        Отсюда $|A - \sigma(x_k, \xi_k)| < \varepsilon$ (для любого разбиения с $d < \delta$)\\
        Таким образом, $A = \lim_{d \to 0} \sigma(x_k, \xi_k)$, т.е. функция $f(x)$ интегрируема
    \end{proof}
    \begin{theorem}
        Для того чтобы ограниченная на сегменте $[a, b]$ функция $f(x)$ была интегрируемой на этом сегменте, необходимо и достаточно, чтобы для любого $\varepsilon > 0$ нашлось такое разбиение $\{x_k\}$ сегмента $[a, b]$, для которого $S - s < \varepsilon$
    \end{theorem}
    \begin{proof}
        
        $\Rightarrow$
        Пусть функция $f(x)$ интегрируема на сегменте $[a, b]$. При доказательстве необходимости вспомогательной теоремы установлено, что для любого $\varepsilon > 0$ существует такое $\delta > 0$, что для любого разюиения сегмента $[a, b]$ с диаметром $d < \delta$, справедливо неарвенство $S-s < \varepsilon$\\
        $\Leftarrow$
        Поскольку
        \begin{gather*}
            s \leq I_* \leq I^* S
        \end{gather*}
        то $I^* - I_* < \varepsilon$. Из этого неравенства и из произвольности $\varepsilon$ заключаем, что $I^* = I_*$, а по вспомогательной теореме получаем, что функция $f(x)$ интегрируема
    \end{proof}
    % -----------------------------------------------------------------------------------



    \subsubsection{Теорема об интегрируемости непрерывной функции на отрезке (с доказательством).}
    \begin{theorem}
        Непрерывные на сегменте $[a, b]$ функции интегрируемы на этом мегменте по Риману
    \end{theorem}
    \begin{proof}
        Пусть $f(x)$ непрерына на сегменте $[a, b]$. Выберем произвольное число $\varepsilon > 0$. Поскольку функция $f(x)$, будучи непрерывной на сегменте, является равномерно непрерывной н нем, то для любого данного $\varepsilon > 0$ существует такое число $\delta > 0$, что если $\xi'$ и $\xi''$ - любые две точки сегмента $[a, b]$, для которых $|\xi' - \xi''| < \delta$, то $|f(\xi') - f(\xi'')| < \frac{\varepsilon}{b-a}$. Отсюда следует, что разность между точными верхней и нижней гранями $f(x)$ на любом сегменте, имеющим длину, меньшую $\delta$, будет меньше числа $\frac{\varepsilon}{b-a}$. Выберем теперь разбиение $\{x_k\}$ сегмента $[a, b]$ с диаметром $d<\delta$. Пусть $M_k = \sup_{x \in [x_{k-1}, x_k]} f(x),\; m_k= \inf_{x \in [x_{k-1}, x_k]} f(x)$\\
        Gо определению верхней и нижней сумм
        \begin{gather*}
            S-s = \sum_{k-1}^n (M_k - m_k) \Delta x_k
        \end{gather*}
        Используя в этом соотношении установленное для выбранного нами разбиения неравенство $M_k - m_k < \frac{\varepsilon}{b-a}$, утверждающее, что разность между точными гранями на любом частичном сегменте меньше $\frac{\varepsilon}{b-a}$, мы получим, что для выбранного разбиения
        \begin{gather*}
            S-s<\frac{\varepsilon}{b-a}\sum_{k-1}^n \Delta x_k = \varepsilon
        \end{gather*}
        ПО основной теореме (критерий интегрируемости по Риману) заключаем, что функция $f(x)$ интегрируема на $[a, b]$
    \end{proof}
    % -----------------------------------------------------------------------------------



    \subsubsection{Теорема об интегрируемости на отрезке функции, имеющей разрывы (с доказательством).}
    \begin{theorem}
        Пусть функция $f(x)$ определена и ограничена на сегменте $[a, b]$. Если для любого числа $\varepsilon>0$ можно указать конечное число интервалов, покрывающих все точки разрыва этой функции и имеющих общую сумму длин, меньшею $\varepsilon$, то $f(x)$ интегрируема по Риману на сегменте $[a, b]$
    \end{theorem}
    \begin{proof}
        Пусть $M$ и $m$ - точные верхняя и нижняя грани функции $f(x)$ постоянна, то, как мы уже доказали, они интегрируема. Поэтому будем считать, что $M>m$. Пусть $\varepsilon > 0$ - произвольное число. Покроем точки разрыва функции $f(x)$ конечным числом интервалов, сумма длин которых не превосходит числа $\varepsilon_1 = \frac{\varepsilon}{2}(M-m)$. Точки сегмента $[a, b]$, не принадлежащие указанным интервалам, очевидно, образуют множество, состоящее из конечного числа непересекающихся сегментов. Назовем эти сегменты дополнительными. На каждом из таких сегментов функция непрерывна, а следовательно, и равномерно непрерывна. Значит, существуют такие числа $\delta_i > 0$, что если $|\xi' - \xi''| < \delta_i$, то $|f(\xi') - f(\xi'') < \frac{\varepsilon}{2(b-a)}|$ для всех $\xi'$ и $\xi''$, принадлежищих $i$-му дополнительному сегменту.\\
        Пусть $\delta = \min_i \delta_i$. Тогда если взять разбиения дополнительных сегментов на частичные сегменты так, чтобы диаметр каждого из частичных сегментов не превосходил $\delta$, то разность между точными верхними гранью $M_k$ и нижней гранью $m_k$ функции $f(x)$ на $k$-м частичном сегменте будет больше $\frac{\varepsilon}{2(b-a)}$. Объединяя все разбиения дополнительных сегментов и указанные выше интервалы с присоединенными к ним концами, мы получим разбиение $\{x_k\}$ всего сегмента $[a, b]$. Для так построенного общего разбиения $[a, b]$. Для так построенного общего разбиения $[a, b]$
        \begin{gather*}
            S-s=\sum_{k=1}^n (M_k-m_k)\Delta x_k = \Sigma' (M_k - m_k) \Delta x_k + \Sigma'' (M_k - m_k) \Delta x_k
        \end{gather*}
        где в $\Sigma'$ перенесены все слагаемые, отвечающие частичным сегментам, образованным из интегралов, покрывающих точки разрыва, а в $\Sigma''$ - все остальные. Рассмотрим первое слагаемое правой части записанного выше равенства. Поскольку $M_k - m_k \leq M - m$ для любого $k$, то $\Sigma' (M_k - m_k) \Delta x_k \leq (M-m) \Sigma' \Delta x_k < \frac{\varepsilon}{2}$\\
        Далее, в силу сказанного выше, из свойства равномерной непрерываности функции $f(x)$ на дополнительных сегментах получаем, что
        \begin{gather*}
            \Sigma'' (M_k - m_k) \Delta x < \frac{\varepsilon}{2(b-a)} \Sigma'' \Delta x_k < \frac{\varepsilon}{2(b-a)}(b-a) = \frac{\varepsilon}{2}
        \end{gather*}
        Таким образом, нами было указано разбиение $\{x_k\}$, для которого $S-s<\varepsilon$. По основной теореме получаем, что функция $f(x)$ интегрируема
    \end{proof}
    % -----------------------------------------------------------------------------------



    \subsubsection{Теорема об интегрируемости функции, монотонной на отрезке (с доказательством).}
    \begin{theorem}
        Монотонная на сегменте $[a, b]$ функция $f(x)$ интегрируема по Риману на этом сегменте
    \end{theorem}
    \begin{proof}
        Случай, когда функция $f(x)$ постоянна на сегменте $[a, b]$, можно исключить. Рассмотрим, например, неубывающую на сегменте $[a, b]$, можно исключить. Рассмотрим, например, неубывающую на сегменте $[a, b]$ функцию $f(x)$. Пусть $\varepsilon > 0$ - произвольное число. Выберем разбиение $\{x_k\}$ сегмента $[a, b]$ с диаметром $d < \frac{\varepsilon}{f(b)-f(a)}$. Заметим, что поскольку $f(x)$ не постоянна, то $f(b) > f(a)$. Оценим разность $S-s = \sum_{k-1}^n (M_k - m_k) \Delta x_k$, $M_k$ и $m_k$ - верхняя и нижняя грани $f(x)$ на $[x_{k-1}, x_k]$. Получим $S - s < \varepsilon \sum_{k-1}^n \frac{M_k - m_k}{f(b) - f(a)}$. Но для неубывающей функции $\sum_{k-1}^n (M_k - m_k) = f(b) - f(a)*$. Поэтому $S - s < \varepsilon$ и функция $f(x)$ интегрируема. Для невозрастающей функции аналогично.
    \end{proof}
    % -----------------------------------------------------------------------------------



    \subsubsection{Теорема об интегрируемости композиции функций на отрезке (с доказательством).}
    % TODO: понять, требуется ли только 9.4, или еще 9.4'
    % -----------------------------------------------------------------------------------



    \subsubsection{Свойства определенного интеграла: интегрируемость суммы, разности, произведения функции и константы, произведения функций (с доказательством).}
    \begin{theorem}
        Пусть функции $f(x)$ и $g(x)$ интегрируемы на сегменте $[a, b]$. Тогда $f(x) \pm f(x)$ также интегрируемы на этом сегменте, причем
        \begin{gather*}
            \int_a^b (f(x) \pm g(x)) dx = \int_a^b f(x) dx \pm \int_a^b g(x) dx
        \end{gather*}
        \label{th:int_prop1}
    \end{theorem}
    \begin{proof}
        При любом разбиении сегмента $[a, b]$ и любом выборе $\xi_k$ справедливо:
        \begin{gather*}
            \sum_{k=1}^n [f(\xi_k) \pm g(\xi_k)] \Delta x_k = \sum_{k = 1}^n f(\xi_k) \Delta x_k \pm \sum_{k=1}^n g(\xi_k) \Delta x_k
        \end{gather*}
        Потому, если существует предел правой части при стремлении диаметра разбиений к нулю, то существует предел и левой части. Из линейных свойств этого предела, которые устанавливаются точно так же, как и для предела последовательностей, вытекает доказанное свойство
    \end{proof}
    \begin{theorem}
        Если функция $f(x)$ интегрируема на сегменте $[a, b]$, то функция $cf(x),\quad c=const$, так же интегрируема на этом сегменте, причем
        \begin{gather*}
            \int_a^b cf(x)dx = c \int_a^b f(x) dx
        \end{gather*}
        \label{th:int_prop2}
    \end{theorem}
    \begin{proof}
        При любом разбиении сегмента $[a, b]$ и любом выборе $\xi_k$ справедливо:
        \begin{gather*}
            \sum_{k=1}^n cf(\xi_k) \Delta x_k = c \sum_{k=1}^n f(\xi_k) \Delta x_k
        \end{gather*}
        Аналогично теореме \ref{th:int_prop1} доказывается и данное свойство 
    \end{proof}
    \begin{theorem}
        Пусть функции $f(x)$ и $g(x)$ интегрируемы на сегменте $[a, b]$. Тогда $f(x)g(x)$ также интегрируема на этом сегменте.
        \label{th:int_prop3}
    \end{theorem}
    \begin{proof}
        \begin{gather*}
            4f(x)g(x) = [f(x) + g(x)]^2 - [f(x) - g(x)]^2
        \end{gather*}
        Заметим, что по теореме 9.4 из интегрируемости какой-либо функции $u(x)$ следует интегрировать ее квадрата. Поскольку $f(x) + g(x)$ и $f(x) - g(x)$ по теореме \ref{th:int_prop1} интегрируемы, то $f(x)g(x)$ тоже интегрируема
        % TODO: почекать ссылку (*)
    \end{proof}
    % -----------------------------------------------------------------------------------



    \subsubsection{Свойства определенного интеграла: интегрируемость на подотрезке, аддитивность (с доказательством).}
    \begin{theorem}
        Пусть функция $f(x)$ интегрируема на сегменте $[a, b]$. Тогда эта функция интегрируема на любом сегменте $[a, b]$. Тогда эта функция интегрируема на любом сегменте $[c, d]$, содержащимся в сегменте $[a, b]$
        \label{th:int_prop4}
    \end{theorem}
    \begin{proof}
        Выберем произвольное число $\varepsilon > 0$ и такое разбиение $\{x_k\}$  сегмента $[a, b]$, что $S - s < \varepsilon$. Добавим к точкам разбиения $\{x_k\}$ точки $c$ и $d$. Для верхних сумм $S'$ и нижних $s'$ вновь полученного разбиения $\{x_k'\}$ в силу леммы 3 $\S$ 2 тоже будет справедлива оценка $S'-s' < \varepsilon$. Рассмотрим разбиение $\{\overline{x}_k\}$ сегмента $[c, d]$, образованное точками разбиения $\{x_k'\}$ всего сегмента $[a, b]$. Для верхних и нижних сумм $\overline{S}$ и $\overline{s}$ разбиения $\{\overline{x}_k\}$ выполнено, очевидно, соотношение $\overline{S}-\overline{s} \leq S'-s'$, поскольку каждое неотрицательное слагаемое $(M_k - m_k) \Delta x_k$ в выражении $\overline{S} - \overline{s}$ будет слагаемым и в выражении $S'-s'$. Таким образом, $\overline{S} - \overline{s} < \varepsilon$, и функция $f(x)$ интегрируема на сегменте $[c, d]$
    \end{proof}
    \begin{proposition}
        Условимся, что
        \begin{gather*}
            \int_a^a f(x) dx = 0
        \end{gather*}
        а
        \begin{gather*}
            \int_a^b f(x) dx = -\int_b^a f(x) dx
        \end{gather*}
        \label{prop:spec_integrals}
    \end{proposition}
    \begin{theorem}
        Если функция $f(x)$ интегрируема на сегментах $[a, c]$ и $[c, b]$, то функция $f(x)$ интегрируема и на сегменте $[a, b]$, причем
        \begin{gather*}
            \int_a^b f(x) dx = \int_a^c f(x) dx + \int_c^b f(x) dx
        \end{gather*}
        \label{th:int_prop5}
    \end{theorem}
    \begin{proof}
        При $a = b$ это свойство справедливо в силу принятых выше соглашений.\\
        Предположим, что $a < c < b$. Выберем произвольное число $\varepsilon > 0$. Пусть $\{x_k'\}$ и $\{x_k''\}$ - такие разбиения сегментов $[a, c]$ и $[c, d]$, что на каждом из этих сегментов $S-s < \frac{\varepsilon}{2}$. Пусть $\{x_k\}$ - разбиение сегмента $[a, b]$, состоящее из точек разбиений $\{x_k'\}$ и $\{x_k''\}$/ Очевидно, что разность между верхней и нижней суммами разбиения $\{x_k\}$ не будет превосходить $\varepsilon$. Интегрируемость функции $f(x)$ на сегменте $[a, b]$ доказана. Пусть теперь $\{x_k\}$ - произвольное разбиение сегмента $[a, b]$, содержащее $c$. Тогда
        \begin{gather*}
            \sum_{k=1}^n f(\xi_k) \Delta x_k = \Sigma' f(\xi_k) \Delta x_k + \Sigma'' f(\xi_k) \Delta x_k
        \end{gather*}
        где $\Sigma'$ берется по частичным сегментам, принадлежащим $[a, c]$, а $\Sigma''$ - по частичным сегментам, принадлежащим $[a, c]$, а $\Sigma''$ - по частичным сегментам, принадлежащим $[c, d]$. Поскольку это верно для любого разбиения, то, перейдя к пределу при стремлении диаметра разбиений к нулю, получим, что
        \begin{gather*}
            \int_a^b f(x) dx = \int_a^c f(x) dx + \int_c^d f(x) dx
        \end{gather*}
        Если $c$ не принадлежит $[a, b]$, то сегмент $[a, b]$ принадлежит либо $[c, b]$, либо $[a, c]$. Пусть, например, $c < a < b$. В силу теоремы \ref{th:int_prop4} функция $f(x)$ интегрируема на $[a, b]$. Дейстительно, функция $f(x)$ интегрируема на $[c, b]$ по условию, а $[a, b] \subset [c, b]$\\
        Далее, поскольку $c < a < b$
        \begin{gather*}
            \int_c^a f(x) dx + \int_a^b f(x) dx = \int_c^b f(x) dx
        \end{gather*}
        Но по утверждению \ref{prop:spec_integrals}, $\int_c^a f(x) dx = -\int_a^c f(x) dx$. Таким образом, свойство доказано для случая, когда $c$ лежит вне сегмента $[a, b]$. Заметим, что формулу этого свойства можно записать как:
        \begin{gather*}
            \int_a^c f(x) dx + \int_c^b f(x) dx + \int_b^a f(x) dx = 0
        \end{gather*}
    \end{proof}
    % -----------------------------------------------------------------------------------



    \subsubsection{Интегрирование неравенств, оценка модуля от интеграла (с доказательством).}
    \begin{theorem}
        Если функции $f(x)$ и $g(x)$ интегрируемы на сегменте $[a, b]$ а $f(x) \leq g(x)$ для всех $x$ из $[a, b]$, то $\int_a^b f(x) dx \leq \int_a^b g(x) dx$
        \label{th:int_ineq}
    \end{theorem}
    \begin{proof}
        Действительно, функция $g(x) - f(x)$ интегрируема и неотрицательна на $[a, b]$, так что
        \begin{gather*}
            \int_a^b (g(x) - f(x)) dx \geq 0
        \end{gather*}
        Но тогда в силу теоремы \ref{th:int_prop1} $\int_a^b g(x) dx - \int_a^b f(x) dx \geq 0$
    \end{proof}
    \begin{theorem}
        Если функция $f(x)$ интегрируема по Риману на сегменте $[a, b]$, то функция $|f(x)|$ интегрируема на этом сегменте и
        \begin{gather*}
            \left| \int_a^b f(x) dx \right| \leq \int_a^b |f(x)| dx
        \end{gather*}
    \end{theorem}
    \begin{proof}
        Рассмотрим функцию $\varphi(t) = |t|$. Согласно теореме 9.4 из интегрируемости $f(x)$ следует интегрируемость $\varphi(f(x)) = |f(x)|$(так как $\varphi(t) = |t|$ на любом сегменте удовлетворяет условию Липшица). Выберем теперь число $\alpha = \pm 1$ так, чтобы $\alpha \int_a^b f(x) dx \geq 0$. Очевидно, что $\alpha f(x) \leq |\alpha f(x)| = |f(x)|$. Тогда в силу теоремы \ref{th:int_ineq}:
        \begin{gather*}
            \left| \int_a^b f(x) dx \right| = \alpha \int_a^b f(x) dx = \int_a^b \alpha f(x) dx \leq \int_a^b |f(x)| dx
        \end{gather*}
        % TODO: ссылка на теорему 9.4
    \end{proof}
    % -----------------------------------------------------------------------------------



    \subsubsection{Первая теорема о среднем значении интеграла (с доказательством).}
    \begin{theorem}
            Пусть каждая из функций $f(x)$ и $g(x)$ интегрируема на сегменте $[a, b]$ и функция $g(x)$, кроме того, неотрицательна(или неположительная) на этом сегменте.\\
            Обозначим через $M$ и $m$ точные грани $f(x)$ на сегменте $[a, b]$. Тогда найдется число $\mu$, удовлетворяющее неравенству $m \leq \mu \leq M$ и такое, что справедлива следующая формула
            \begin{gather}
                \int_a^b f(x)g(x) dx = \mu \int_a^b g(x) dx \label{eq:mean_formula1}
            \end{gather}
            При дополнительном предположении о непрерывности $f(x)$ на сегменте $[a, b]$ можно утверждать на этом сегменте точка $\xi$ такая, что справедлива формула
            \begin{gather}
                \int_a^b f(x)g(x) dx = f(\xi) \int_a^b g(x) dx \label{eq:mean_formula1_func}
            \end{gather}
            Формулу \ref{eq:mean_formula1_func} принято называть \textbf{первой формулой среднего значения}. Формулу \ref{eq:mean_formula1} иногда тоже так называют.
    \end{theorem}
    \begin{proof}
        % TODO: огромное сложное доказательство, мне страшна
    \end{proof}
    \begin{corollary}
        При формулировке теоремы с $g(x) = 1$ можно утверждать, что
        \begin{gather*}
            \int_a^b f(x) dx = \mu (b-a)
        \end{gather*}
        и, соответственно,
        \begin{gather*}
            \int_a^b f(x) dx = f(\xi)(b-a)
        \end{gather*}
    \end{corollary}
    % -----------------------------------------------------------------------------------



    \subsubsection{Определение интеграла с переменным верхним пределом. Теорема: непрерывность интеграла с переменным верхним пределом (с доказательством).}
    \begin{definition}
        Пусть $f(x)$ непрерывана на $[a, b]$. Пусть $p$ принадлежит $[a, b]$. Тогда $\forall x \in [a, b]$ функция $f(x)$ интегрируема на $[p, x] \Rightarrow$ на $[a, b]$ определена функция $F(x)=\int_p^x f(t) dt$, которая называется \textbf{интегралом с переменным верхним пределом}
    \end{definition}
    \begin{theorem}
        Если функция $f(x)$ интегрируема на сегменте $[a, b]$, $p$ - любая точка этого сегмента, то производная функции $F(x) = \int_p^x f(t) dt$ существует в каждой точке $x_0$ непрерывности подынтегральной функции, причем $F'(x_0) = f(x_0)$
    \end{theorem}
    \begin{proof}
        В силу непрерывности функции $f(x)$ в точке $x_0$ для любого $\varepsilon > 0$ найдется такое $\delta$, что $f(x_0) - \varepsilon < f(x) < f(x_0) + \varepsilon$, если $|x - x_0| < \delta$. для всех $t$ из $[x_0, x]$ выполняется неравенство $|t- x_0| \leq |x - x_0| < \delta$. Поэтому для всех таких $t$
        \begin{gather*}
            f(x_0) - \varepsilon \leq f(t) \leq f(x_0) + \varepsilon
        \end{gather*}
        Согласно первой формуле среднего значения(формула \ref{eq:mean_formula1}) полчуим из этих неравенств
        \begin{gather*}
            f(x_0) - \varepsilon \leq \frac{1}{x - x_0} \int_{x_0}^x f(t) dt \leq f(x_0) + \varepsilon \text{ при }|x - x_0| < \delta
        \end{gather*}
        (Значение $\mu = \frac{1}{x - x_0} \int_{x_0}^x f(t) dt$ не меняется при перестанвке $x$ и $x_0$). Но $\frac{1}{x - x_0} \int_{x_0}^x f(t) dt = \frac{F(x) - F(x_0)}{x - x_0}$, следовательно при $|x - x_0| < \delta$
        \begin{gather*}
            f(x_0) - \varepsilon \leq \frac{F(x) - F(x_0)}{x - x_0} \leq f(x_0) + \varepsilon
        \end{gather*}
        т.е. $F'(x)$ существует и равна $f(x_0)$
    \end{proof}
    % -----------------------------------------------------------------------------------



    \subsubsection{Определение интеграла с переменным верхним пределом. Теорема: дифференцируемость интеграла с переменным верхним пределом (с доказательством).}
    \begin{definition}
        Пусть $f(x)$ непрерывана на $[a, b]$. Пусть $p$ принадлежит $[a, b]$. Тогда $\forall x \in [a, b]$ функция $f(x)$ интегрируема на $[p, x] \Rightarrow$ на $[a, b]$ определена функция $F(x)=\int_p^x f(t) dt$, которая называется \textbf{интегралом с переменным верхним пределом}
    \end{definition}
    \begin{theorem}
        % who блять?
    \end{theorem}
    % -----------------------------------------------------------------------------------
    


    \subsubsection{Определение интеграла с переменным верхним пределом. Теорема: формула Ньютона-Лейбница (с доказательством).}
    \begin{definition}
        Пусть $f(x)$ непрерывана на $[a, b]$. Пусть $p$ принадлежит $[a, b]$. Тогда $\forall x \in [a, b]$ функция $f(x)$ интегрируема на $[p, x] \Rightarrow$ на $[a, b]$ определена функция $F(x)=\int_p^x f(t) dt$, которая называется \textbf{интегралом с переменным верхним пределом}
    \end{definition}
    \begin{theorem}[формула Ньютона-Лейбница]
        Для того чтобы вычислить определенный интеграл по сегменту $[a, b]$ от непрерывной функции $f(x)$, следует вычислить значение произвольной ее первообразной в точке $b$ и в точке $a$ и вычесть из первого значения второе.
    \end{theorem}
    \begin{proof}
        Две первообразные одной функции отличаются на константу. Поэтому пусть $F(x) = \int_a^x f(t) dt$, а $\Phi(x)$ - любая другая первообразная непрерывной функции $f(x)$, то $\Phi(x) - F(x) = C = const$, т.е. $\Phi(x) = \int_a^x f(t) dt + C$(см. теорему 9.5). Положим в посденей формуле сначала $x = a$, а затем $x = b$. Как мы условились(утверждение \ref{prop:spec_integrals}), $\int_a^a f(t) dt = 0$ для любой функции, принимающей конечное значение в точке $a$, поэтому $\Phi(a) = C,\quad \Phi(b) = \int_a^b f(x) dx + C$. Отсюда
        \begin{gather*}
            \int_a^b f(x) dx = \Phi(x)|^b_a = \Phi(b) - \Phi(a)
        \end{gather*} 
    \end{proof}
    % -----------------------------------------------------------------------------------



    \subsubsection{Формулы замены переменной и интегрирования по частям в определенном интеграле (с доказательством).}
    \begin{theorem}
        Пусть функция $x = g(t)$ имеет непрерывную произвудную на сегменте $[m, M]$ и $\min_{t \in [m, M]} g(t) = a, \quad \max_{t \in [m, M]} g(t) = b$, причем $g(m) = a, \quad g(M) = b$. Тогда $\int_a^b f(x) dx = \int_m^M f[g(t)]g'(t) dt$(при условии, что функция $f(x)$ непрерывна на сегменте $[a, b]$)\\
        Указанна формула называется \textbf{формулой замены переменной} под знаком определенного интеграла.
    \end{theorem}
    \begin{proof}
        Пусть $\Phi(x)$ - некоторая первообразная функции $f(x)$. Функции $\Phi(x)$ и $x = g(t)$ дифференцируемы на сегментах $[a, b]$ и $[m, M]$ соответственно. Поэтому, согласно правилу вычисления производной сложной функции, для всех $t$ из $[m, M]$
        \begin{gather*}
            \frac{d}{dt} \Phi(g(t)) = \Phi'(g(t))g'(t)
        \end{gather*}
        Заметим, что производная $\Phi'$ в выражении справа вычислена по аргументу $x$: $\Phi'(g(t)) = \Phi'(x), \quad x = g(t)$\\
        Заметим также, что $\Phi'(x) = f(x)$. Подставив в правую часть формулы для $\frac{d}{dt}\Phi(g(t))$ это равенство, получаем
        \begin{gather*}
            \frac{d}{dt}\Phi(g(t)) = f(g(t))g'(t)
        \end{gather*}
        Таким образом, функция $\Phi(g(t))$ является на сегменте $m \leq t \leq M$ первообразной для функции $f(g(t))g'(t)$, т.е.
        \begin{gather*}
            \int_m^M f(g(t))g'(t) dt = \Phi(g(M)) - \Phi(g(m)) = \Phi(b) - \Phi(a)
        \end{gather*}
        Согласно условию. Следовательно, с одной стороны, $\int_a^b f(x) dx = \Phi(b) - \Phi(a)$, а, с другой стороны, $\Phi(b) - \Phi(a) = \int_m^M f(g(t))g'(t) dt$, что и требовалось
    \end{proof}
    \begin{theorem}[правило интегрирования по частям]
        Пусть функции $f(x)$ и $g(x)$ имеют непрерывные производные на сегменте $[a, b]$, тогда
        \begin{gather*}
            \int_a^b f(x)g'(x) dx = f(x)g(x)|_a^b - \int_a^b g(x)f'(x) dx
        \end{gather*}
    \end{theorem}
    \begin{proof}
        Действительно, $\frac{d}{dx}[f(x)g(x)] = f(x)g'(x) + f'(x)g(x)$. Поэтому функция $f(x)g(x)$ является первообразной функции $[f(x)g'(x) + f'(x)g(x)]$. Следовательно,
        \begin{gather*}
            \int_a^b [f(x)g'(x) + f'(x)g(x)]dx = f(x)g(x)|_a^b
        \end{gather*}
    \end{proof}
    % -----------------------------------------------------------------------------------



    \subsection{Геометрические приложения определенного интеграла}
    \subsubsection{Определение длины плоской кривой. Теорема о спрямляемости и длине дуги кривой, заданной параметрически (с доказательством).}
    \begin{definition}
        % TODO: определение длины кривой
    \end{definition}
    \begin{theorem}
        Пусть функции $x = \varphi(t)$ и $y = \psi(t)$ непрерывна и имеют непрерывные первые производные на сегменте $[\alpha, \beta]$. Тогда кривая $L$, определяемая параметрически уравнениями $x = \varphi(t), \quad y = \psi(t)$ при $t$ из $[\alpha, \beta]$, спрямляема и длина $|L|$ ее дуги может быть вычислена по формуле
        \begin{gather*}
            |L| = \int_{\alpha}^{\beta} \sqrt{\varphi'^2(t) + \psi'^2(t)} dt
        \end{gather*}
    \end{theorem}
    \begin{proof}
        Сначала докажем, что кривая $L$ спрямляема. Рассмотрим формулу для длин $|l|$ ломаной $l$, вписанной в кривую $[\alpha, \beta]$:
        \begin{gather*}
            |l| = \sum_{i = 1}^n \sqrt{[\varphi(t_i) - \varphi(t_{i-1})]^2 + [\psi(t_i) - \psi(t_{i-1})]^2}
        \end{gather*}
        Для каждой из функций $\varphi(t)$ и $\psi(t)$ выполнены на каждом частичном сегменте $[t_{i-1}, t_i]$ (при $i=1, 2, \ldots, n$) все условия теоремы 6.4 Лагранжа. В силу этой теоремы меджу $t_{i-1}$ и $t_i$ найдутся точки такие, что будут справедливы равенства
        \begin{gather*}
            \varphi(t_i) - \varphi(t_{i-1}) = \varphi'(\xi_i) \Delta t_i,\\
            \psi(t_i) - \psi(t_{i-1}) = \psi'(\mu_i) \Delta t_i
        \end{gather*}
        где $\Delta t_i = t_i - t_{i-1}$\\
        Следовательно,
        \begin{gather}
            |l| = \sum_{i-1}^n \sqrt{\varphi'^2(\xi_i) + \psi'^2(\mu_i)} \Delta t_i \label{eq:dist_sum}
        \end{gather}
        По условию теоремы функции $\varphi(t)$ и $\psi(t)$ имеют на сегменте $[\alpha, \beta]$ непрерывные, а потому и ограниченные первые производные, т.е. для всех $t$, лежащих внутри сегмента $[\alpha, \beta]$, справедливы неравенства $|\varphi'(t)| \leq M, \quad |\psi'(t)| \leq M$. Поэтому из формулы \ref{eq:dist_sum} следует, что
        \begin{gather*}
            0 < |l| \leq \sum_{i = 1}^n \sqrt{M^2 + M^2} \Delta t_i = M \sqrt{2} \sum_{i = 1}^n \Delta t_i = M \sqrt{2} (\beta - \alpha)
        \end{gather*}
        Таким образом, множество $\{|l|\}$ длин вписанных в кривую $L$ ломанных, отвечающих всевозможным разбиениям $T$ сегмента $[\alpha, \beta]$, ограничено, и по определению кривая $L$ спрямляема.
        \begin{remark}
            хз, надо ли тут доказывать формулу, но доказательство пиздец
        \end{remark}
    \end{proof}
    % -----------------------------------------------------------------------------------


    
    \subsubsection{Определение площади плоской фигуры. Два критерия квадрируемости (через приближение простейшими и квадрируемыми (с доказательством).}
    \begin{definition}
        % TODO: определение площади многоугольных фигур
    \end{definition}
    \begin{theorem}
        \label{th:quadracity_crit}
        Для квадрируемости плоской фигуры $F$ необходимо и достаточно, чтобы для любого $\varepsilon > 0$ нашлись такая описанная вокруг $F$ многоугольная фигура $Q$ и такая вписанная в $F$ многоугольная фигура $P$, для которых
        \begin{gather*}
            \mu(Q) - \mu(P) < \varepsilon
        \end{gather*}
    \end{theorem}
    \begin{proof}
        $\Rightarrow$
        Пусть фигура $F$ квадрируема, т.е. $\mu_*=\mu^*$. По определению точных граней $\mu_*=\sup_{P \subset F} \mu(P)$ и $\mu^*=\inf_{Q \supset F} \mu(Q)$ для любого фиксированного нами $\varepsilon > 0$ найдутся вписанная многоугольная фигура $P$ и описанная многоугольная фигура $Q$ такие, что
        \begin{gather*}
            \mu_* - \frac{\varepsilon}{2} < \mu(P) \leq \mu_*, \quad \mu^* < \mu(Q) \leq \mu^* + \frac{\varepsilon}{2}
        \end{gather*}
        Из этих равенств и из равенства $\mu_* = \mu^*$ заключаем, что $\mu(Q) - \mu(P) < \varepsilon$\\
        $\Leftarrow$
        ПУсть для любого $\varepsilon > 0$ существуют многоугольные фигуры $P$ и $Q$, указанные в формулировке теоремы. Тогда из неравенства $\mu(Q) - \mu(P) < \varepsilon$ и из соотношений
        \begin{gather*}
            \mu(P) \leq \mu_* \leq \mu^* \leq \mu(Q)
        \end{gather*}
        получаем, что
        \begin{gather*}
            0 \leq \mu^* - \mu_* \leq \mu(Q) - \mu(P) < \varepsilon
        \end{gather*}
        В силу произвольности $\varepsilon > 0$, то из условия $0 \leq \mu^* - \mu_* < \varepsilon$ вытекает, что $\mu^* = \mu_*$, т.е. фигура $F$ квадрируема
    \end{proof}
    \begin{theorem}
        Для квадрируемости плоской фигуры $F$ необходимо и достаточно, чтобы для любого $\varepsilon > 0$ нашлись такие содержащие $F$ квадрируемая плоская фигура $Q$ и такая содержащаяся в $F$ квадрирумая плоская фигура $P$ для которых
        \begin{gather*}
            \mu(Q) - \mu(P) < \varepsilon
        \end{gather*}
    \end{theorem}
    \begin{proof}
        $\Rightarrow$
        Многоугольные фигуры $P$ и $Q$ всегда квадрируемы\\
        $\Leftarrow$
        Фиксируем произвольное $\varepsilon > 0$ и построим по нему квадрируемы плоские фигуры $Q$ и $P$, первая из которых содержит $F$, а вторая содержится в $F$, так, что
        \begin{gather*}
            \mu(Q) - \mu(P) < \frac{\varepsilon}{2}
        \end{gather*}
        Так как $Q$ и $P$ - квадрируемые плоские фигуры, то найдется многоугольная фигура $Q$, содержащая $Q$, и многоугольная фигура $\hat{Q}$, содержащая $Q$, и многоугольная фигура $\hat{P}$, содержащая $P$, такие, что
        \begin{gather*}
            \mu(\hat{Q}) - \mu(Q) < \frac{\varepsilon}{4},\; \mu(P) - \mu(\hat{P}) < \frac{\varepsilon}{4}
        \end{gather*}
        Из трех последних неравенств вытекает, что $\mu(\hat{Q}) - \mu(\hat{P}) < \varepsilon$. Но тогда, поскольку многоугольная фигура $\hat{Q}$ содержит $F$, а многоугольная фигура $\hat{P}$, содержится в $F$, фигура $F$, фигура $F$ квадрируема в силу предыдущей теоремы 
    \end{proof}
    % -----------------------------------------------------------------------------------



    \subsubsection{Определение площади плоской фигуры. Теорема о квадрируемости и площади криволинейной трапеции (с доказательством).}
    \begin{definition}
        % TODO: определение площади многоугольных фигур
    \end{definition}
    \begin{corollary}
        Криволинейная трапеция представляет собой квадрируемую фигуру $F$, площадь которой $\mu(F)$ вычисляется по формуле
        \begin{gather}
            \mu(F) = \int_{a}^{b} f(x) dx \label{eq:curved_trap}
        \end{gather}
    \end{corollary}
    \begin{proof}
        Непрерывная на сегменте $[a, b]$ функция $f(x)$ интегрируема, поэтому для любого положительного числа $\varepsilon$ можно указать такое разбиение сегмента $[a, b]$, для которого разность между верхней суммой $S$ и нижней суммой $s$ будет меньше $\varepsilon$. Но $S=\mu(Q)$ и $s = \mu(P)$, где $\mu(Q)$ и $\mu(P)$ - площади многоугольных фигур, первая из которых содержит криволинейную трапецию, а вторая содержится в криволинейной трапеции. Таким образом, $\mu(Q) - \mu(P) < \varepsilon$, и в силу теоремы \ref{th:quadracity_crit} криволинейная трапеция квадрируема. Поскольку для любой интегрируемой функции предел при стремлении диаметра разбиения к нулю как верхних $S$, так и нижних сумм $s$ равен $\int_{a}^{b} f(x) dx$ и $s \leq \mu(F) \leq S$, то площадь $\mu(F)$ криволинейной трапеции находится по формуле \ref{eq:curved_trap}
    \end{proof}
    % -----------------------------------------------------------------------------------



    \subsubsection{Определение площади плоской фигуры. Теорема о квадрируемости и площади криволинейного сектора (с доказательством).}
    \begin{definition}
        % TODO: определение площади многоугольных фигур
    \end{definition}
    \begin{corollary}
        Криволинейный сектор предоставляет собой квадрируемую фигуру $F$, площадь $\mu(F)$ которой может быть вычислена по формуле
        \begin{gather*}
            \mu(F) = \frac{1}{2} \int_{\alpha}^{\beta} r^2 (\theta) d \theta \label{eq:curved_sec}
        \end{gather*}
    \end{corollary}
    \begin{proof}
        Рассмотрим разбиение сегмента $[\alpha, \beta]$ точками $\alpha = \theta_0 < \theta_1 < \ldots < \theta_n = \beta$ и для каждого частичного сегмента $[\theta_{i-1}, \theta_i]$ посмотрим круговые секторы, радиусы которых равны минимальному $r_i$ и максимальному $R_i$ значениям функции $r(\theta)$ на сегменте $[\theta_{i-1}, \theta_i]$. В результате получатся две квадрируемые фигуры, первая фигура $A$ содержится в криволинейном секторе, а вторая $B$ содержит этот сектор.\\
        Площади $\mu(A)$ и $\mu(B)$ указанных квадрируемых фигур $A$ и $B$ соответственно равны $\frac{1}{2} \sum_{i=1}^n r_i^2 (\theta_i - \theta_{i-1})$ и $\frac{1}{2} \sum_{i-1}^n R_i^2(\theta_i - \theta_{i-1})$\\
        Обратим внимание на то, что первая из этих сумм является нижней суммой $s$, а вторая - верхней суммой $S$ функции $\frac{1}{2} r^2 (\theta)$ на сегменте $[\alpha, \beta]$ для указанного разбиения этого сегмента. Так как непрерывная на $[\alpha, \beta]$ функция $\frac{1}{2} r^2 (\theta)$ интегрируема на этом сегменте, то для любого $\varepsilon > 0$ найдется разбиение, для которого разность $S - s = \mu(B) - \mu(A)$ меньше $\varepsilon$.\\
        Так как $A$ и $B$ - две квадрируемые фигуры, первая из которых содержится в криволинейном секторе $F$, а вторая содержит $F$, то в силу критерия квадрируемости криволинейный сектор квадрируем.\\
        Справедливость для его площади формулы \ref{eq:curved_sec} вытекает из того, что эта площадь $\mu(F)$ заключена между $s = \mu(A)$ и $S = \mu(B)$, а обе суммы $s$ и $S$ стремятся к интегралу, стоящему в правой части \ref{eq:curved_sec}, при стремлении диаметра разбиения к нулю
    \end{proof}
    % -----------------------------------------------------------------------------------



    \subsubsection{Определение объема тела в пространстве. Два критерия кубируемости (через приближение простейшими и кубируемыми, с доказательством).}
    \begin{definition}
        % TODO: определение объема нужна!
    \end{definition}
    \begin{theorem}
        Для кубируемости тела $F$ необходимо и достаточно того, чтобы для любого $\varepsilon > 0$ нашлись такое содержащиеся в $F$ многогранное тело $P$ и такое содержащее $F$ многогранное тело $Q$, для которых $\mu(Q) - \mu(P) < \varepsilon$
        \label{th:qubic_crit}
    \end{theorem}
    \begin{proof}
        Аналогично \ref{th:quadracity_crit}
    \end{proof}
    \begin{theorem}
        Тело $F$ кубируемо тогда и только тогда, когда его граница имеет объем нуль
    \end{theorem}
    % TODO: а шо делац с доказательством?
    % -----------------------------------------------------------------------------------



    \subsubsection{Определение объема тела в пространстве. Кубируемость цилиндрического тела и ступенчатого тела (с доказательством).}
    \begin{definition}
        % TODO: определение объема нужна!
    \end{definition}
    \begin{corollary}
        Если основанием цилиндрического тела $F$ является плоская квадрируемая фигура $G$, то тело $F$ кубируемо, причем объем $\mu(F)$ этого тела равен $\mu(G)h$, где $\mu(G)$ - площадь основания $G$, а $h$ - высота этого цилиндрического тела.
    \end{corollary}
    \begin{proof}
        Поскольку плоская фигура $G$ квадрируема, то для любого $\varepsilon > 0$ можно указать такие описанную и вписанную в эту фигуру многоугольные фигуры $Q$ и $P$, что $\mu(Q) - \mu(P) < \frac{\varepsilon}{h}$\\
        Объемы цилиндрических многогранных тел $F_Q$ и $F_P$, основанием которых служат многоугольные фигуры $Q$ и $P$, а высота которых равна $h$, равны соответственно $\mu(Q)h$ и $\mu(P)h$. Поэтому
        \begin{gather*}
            \mu(Q)h - \mu(P)h = \left[ \mu(Q) - \mu(P) \right] h < \frac{\varepsilon}{h} h = \varepsilon
        \end{gather*}
        Так как многогранное тело $F_Q$ содержит $F$, а многогранное тело $F_P$ содержится в $F$, то в силу теоремы \ref{th:qubic_crit} тело $F$ кубируемо. Поскольку $\mu(P)h \leq \mu(G)h \leq \mu(Q)h$, то объем цилиндрического тела $F$ равен $\mu(G)h$
    \end{proof}
    % TODO: ну как там со ступенчатыми телами?
    % -----------------------------------------------------------------------------------



    \subsubsection{Определение объема тела в пространстве. Кубируемость тела, образованного вращением криволинейной трапеции вокруг оси Ox (с доказательством).}
    \begin{definition}
        % TODO: определение объема нужна!
    \end{definition}
    \begin{corollary}
        Пусть функция $y = f(x)$ непрерывна на сегменте $[a, b]$. Тогда тело $F$, образованное вращением вокруг оси $Ox$ криволинейной трапеции, ограниченной графиком функции $|f(x)|$ ординатами в точках $a$ и $b$ и отрезком оси $Ox$ от $a$ до $b$, кубируемо и его объем $\mu(F)$ может быть найден по формуле
        \begin{gather}
            \mu(F) = \pi \int_{a}^{b} f^2(x) dx \label{eq:rot_vol}
        \end{gather}
    \end{corollary}
    \begin{proof}
        Разобьем сегмент $[a, b]$ на частичные сегментами точками $a = x_0 < x_1 < \ldots < x_n = b$. Пусть $m_i$ и $M_i$ - точные грани $f(x)$ на частичном сегменте $[x_{i-1}, x_i]$. На каждом таком сегменте построим два прямоугольника с высотами $m_i$ и $M_i$. В результате получается две ступенчатые фигуры, одна из которых в криволинейной трапеции, а другая содержит ее. При вращении криволинейной трапеции и этих ступенчатых мы получим тело $F$ и два ступенчатых тела, одно из которых $Q$ содержит $F$, а другое $P$ содержится в $Q$. Объемы этих тел $Q$ и $P$ равны соответственно
        \begin{gather*}
            \mu(Q) = \pi \sum_{i=1}^n M^2_i \Delta x_i,\quad \mu(P) = \pi \sum_{i=1}^n m^2_i \Delta x_i
        \end{gather*}
        Легко увидеть, что эти выражения представляют собой верхнюю и нижнюю суммы для функции $\pi f^2(x)$. Поскольку эта функция интегрируема, то разность указанных сумм для некоторого разбиения сегмента $[a, b]$ будет меньше наперед взятого положительного числа $\varepsilon$. Следовательно, тело кубируемо. Поскольку предел указанных сумм при стремлении диаметра разбиения сегмента $[a, b]$ к нулю равен $\pi \int_{a}^{b} f^2(x) dx$, то объем $\mu(F)$ тела $F$ вычисляется по формуле \ref{eq:rot_vol} 
    \end{proof}
    % -----------------------------------------------------------------------------------


    
    \subsection{Несобственные интегралы}
    \subsubsection{Определение несобственного интеграла первого рода. Теоремы: замена переменной и интегрирование по частям в несобственном интеграле первого рода (с доказательствами).}
    \begin{definition}
        Предел
        \begin{gather*}
            \lim_{A \to + \infty} F(A) = \lim_{A \to +\infty}\int_{a}^{A} f(x) dx
        \end{gather*}
        d случае, если он существует, называется \textbf{несобственным интегралом первого рода} от функции $f(x)$ по полупрямой $[a, +\infty)$ и обозначается символом
        \begin{gather*}
            \int_{a}^{+\infty} f(x) dx
        \end{gather*}
        \begin{remark}
            Аналогично определяются несобственные интегралы по полупрямой $-\infty < x \leq b$ и по всей бесконечной прямой $-\infty < x < +\infty$
            Определяются они как:
            \begin{gather*}
                \int_{-\infty}^{b} f(x) dx := \lim_{A \to -\infty} \int_{A}^{b} f(x) dx\\
                \int_{-\infty}^{+\infty} f(x) dx = \lim_{A' \to -\infty,\; A'' \to +\infty} \int_{A'}^{A''} f(x) dx
            \end{gather*}
        \end{remark}
    \end{definition}
    % TODO: шо с теоремами?(стр 379 исс)
    % -----------------------------------------------------------------------------------



    \subsubsection{Критерий Коши и признак сравнения сходимости несобственного интеграла первого рода (с доказательствами).}
    \begin{corollary}[критерий Коши сходимости несобственного интеграла]
        Для сходимости несобственного интеграла $\int_{a}^{+\infty} f(x) dx$ необходимо и достаточно, чтобы для любых $A_1$ и $A_2$, превосходящих $B$,
        \begin{gather*}
            \left| \int_{A_1}^{A_2} f(x) dx \right| < \varepsilon
        \end{gather*}
    \end{corollary}
    \begin{proof}
        Условие данного утверждения эквивалентно условию существованию предельного значения функции $F(A) = \int_{0}^{A} f(x) dx$ при $A \to +\infty$. Для существования такого предела необходим и достаточно чтобы функция удовлетворяла условию Коши: для любого $\varepsilon>0$ можно указать такое $B>a$, что для любых $A_1$ и $A_2$, превосходящих $B$, выполняется равенство
        \begin{gather*}
            |F(A_2) - F(A_1)| = \left| \int_{A_1}^{A_2} f(x) dx \right| < \varepsilon
        \end{gather*} 
    \end{proof}
    % TODO: ААААААААА ЧТО ПРОИСХОДИТ
    % -----------------------------------------------------------------------------------



    \subsubsection{Определение абсолютной и условной сходимости несобственного интеграла первого рода. Признак Дирихле-Абеля сходимости несобственного интеграла первого рода (с доказательством).}
    \begin{definition}
        Несобственный интеграл $\int_{a}^{+\infty} f(x) dx$ называется \textbf{абсолютно сходящимся}, если сходится $\int_{a}^{+\infty} |f(x)|dx$
    \end{definition}
    \begin{definition}
        Несобственный интеграл $\int_{a}^{+\infty} f(x) dx$ называется \textbf{условно сходящимся}, если он сходится, а интеграл $\int_{a}^{+\infty} f(x) dx$ расходится 
    \end{definition}
    \begin{theorem}[признак Дирихле-Абеля]
        Пусть выполнены следующие три условия:
        \begin{enumerate}
            \item функция $f(x)$ непрерывна на полупрямой $a \leq x < +\infty$ и имеет на этой полупрямой ограниченную первообразную $F(x)$
            \item функция $g(x)$ определена и монотонно не возрастает на полупрямой $a \leq x < +\infty$ и имеет равный нулю предел при $x \to +\infty$
            \item производная $g'(x)$ функции $g(x)$ существует и непрерывна в каждой точке полупрямой $a \leq x < +\infty$. При выполнении этих трех условий несобственный интеграл
            \begin{gather}
                \int_{a}^{+\infty} f(x) g(x) dx \label{eq:DA_th}
            \end{gather}
            сходится
        \end{enumerate}
    \end{theorem}
    \begin{proof}
        Воспользуемся критерием Коши сходимости несобственных интегралов. Предварительно проведем интегрирование по частям интеграла $\int_{A_1}^{A_2} f(x) dx$ на произвольном сегменте $[A_1, A_2],\quad A_2>A_1$ полупрямой $a \leq x < +\infty$. Получим
        \begin{gather}
            \int_{A_1}^{A_2} f(x)g(x)dx = F(x)g(x)|^{A_2}_{A_1} - \int_{A_1}^{A_2}F(x)g'(x)dx \label{eq:DA_pr_1}
        \end{gather}
        По условию теоремы $F(x)$ ограничена: $|F(x)| \leq K$. Так как $g(x)$ не возрастает и стремится к нулю при $x \to +\infty$, то $g(x) \leq 0$, а $g'(x) \leq 0$. Таким образом, оценивая \ref{eq:DA_pr_1}, получим
        \begin{gather*}
            \left| \int_{A_1}^{A_2} f(x)g(x)dx \right| \leq K[g(A_1) + g(A_2)] + K \int_{A_1}^{A_2}(-g'(x))dx
        \end{gather*}
        Так как интеграл в правой части этого неравенства равен $g(A_1) - g(A_2)$, то, очевидно,
        \begin{gather}
            \left| \int_{A_1}^{A_2} f(x)g(x) dx \right| \label{eq:DA_pr_3}
        \end{gather}
        Используя это неравенство, нетрудно завершить доказательство теоремы. Пусть $\varepsilon$ - произвольное положительное число. Так как $g(x) \in 0$ при $x \to +\infty$, то по данному $\varepsilon$ можно выбрать $B$ так, что при $A_1 \geq B$ выполняется неравенство $g(A_1) < \frac{\varepsilon}{2K}$. Отсюда и из неравенства \ref{eq:DA_pr_3} следует, что $\forall A_1, A_2 > B$, выполняется неравенство $\left| \int_{A_1}^{A_2} f(x)g(x)dx \right| < \varepsilon$, которое, согласно критерию Коши, гаранитрует сходимость интеграла \ref{eq:DA_th}
    \end{proof}
    % -----------------------------------------------------------------------------------



    \subsubsection{Определение несобственного интеграла первого и второго рода. Понятие главного значения несобственного интеграла первого и второго рода. Главное значение интеграла первого рода для нечетной функции.}
    \begin{definition}
        Предел
        \begin{gather*}
            \lim_{A \to + \infty} F(A) = \lim_{A \to +\infty}\int_{a}^{A} f(x) dx
        \end{gather*}
        d случае, если он существует, называется \textbf{несобственным интегралом первого рода} от функции $f(x)$ по полупрямой $[a, +\infty)$ и обозначается символом
        \begin{gather*}
            \int_{a}^{+\infty} f(x) dx
        \end{gather*}
        \begin{remark}
            Аналогично определяются несобственные интегралы по полупрямой $-\infty < x \leq b$ и по всей бесконечной прямой $-\infty < x < +\infty$
            Определяются они как:
            \begin{gather*}
                \int_{-\infty}^{b} f(x) dx := \lim_{A \to -\infty} \int_{A}^{b} f(x) dx\\
                \int_{-\infty}^{+\infty} f(x) dx = \lim_{A' \to -\infty,\; A'' \to +\infty} \int_{A'}^{A''} f(x) dx
            \end{gather*}
        \end{remark}
    \end{definition}
    \begin{definition}
        Правый предел
        \begin{gather*}
            \lim_{\alpha \to +0} \int_{a}^{b - \alpha} f(x)dx
        \end{gather*}
        в случае, если он существует, называется \textbf{несобственным интегралом второго рода} от функции $f(x)$ по сегменту $[a, b]$ и обозначается символом
        \begin{gather*}
            \int_{a}^{b} f(x) dx
        \end{gather*}
    \end{definition}
    \begin{definition}
        Главным значением несобственного интеграла от функции $f(x)$ мы будеи называеть предел
        \begin{gather*}
            V.p. \int_{-\infty}^{+\infty} f(x) dx := \lim_{A \to +\infty} \int_{-A}^{A} f(x) dx
        \end{gather*}
    \end{definition}
    % TODO: а шо дэлац?
    % -----------------------------------------------------------------------------------



    \subsubsection{Метод хорд нахождения корня уравнения и его обоснование.}
    % -----------------------------------------------------------------------------------



    \subsubsection{Метод касательных нахождения корня уравнения и его обоснование.}
    % -----------------------------------------------------------------------------------



    \subsubsection{Метод прямоугольников вычисления определенного интеграла (с выводом оценки погрешности).}
    % -----------------------------------------------------------------------------------



    \subsubsection{Формула Тейлора с остаточным членом в интегральной форме (с доказательством).}
    % -----------------------------------------------------------------------------------



    \subsection{Функции многих переменных}
    \subsubsection{Определение предела последовательности в пространстве \texorpdfstring{$\mathbb{R}_m$}{Rm}. Леммы об эквивалентности сходимости последовательности в пространстве \texorpdfstring{$\mathbb{R}_m$}{Rm} и покоординатной сходимости, фундаментальности в пространстве \texorpdfstring{$\mathbb{R}_m$}{Rm} и покоординатной фундаментальности (с доказательствами).}
    \begin{definition}
        Последовательность $\{M_n\}$ nочек евклидова пространства $E^m$ называется \textbf{сходящейся}, если существует точка $A$ просторанства $E^m$ такая, что для любого положительного числа $\varepsilon$ можно указать отвечающий ему номер $N$ такой, что при $n \geq N$ выполняется неравенство $\rho(M_n, A) < \varepsilon$. При этом точка $A$ называется пределом последовательности $\{M_n\}$\\
        Обозначение: $\lim_{n \to \infty} M_n = A, \text{ или } M_n \to A \text{ при } n \to \infty$
    \end{definition}
    \begin{lemma}
        Последовательность $\{M_n\}$ точек $m$-мерного евклидового пространства $E^m$ сходится к точке $A$ этого пространства тогда и только тогда, когда числовые последовательности $\{x_1^{(n)}\}$, $\{x_2^{(n)}\}$, \ldots, $\{x_m^{(n)}\}$ координат точек $M_n$ сходятся соответственно к числам $a_1, a_2, \ldots, a_m$, представляющим собой координаты точки $A$
    \end{lemma}
    \begin{proof}
        % TODO: AAAAAAAAAAAAAAAAAAAAAAAAAAAAAAAAAAAAAAAAAAAAAAAAAAAAAAAAAAAAAAAAAAAAAAAA
    \end{proof}
    \begin{lemma}
        Последовательность $\{M_n\}$ точек $m$-мерного евклидова пространства $E^m$ является фундаментальной тогда и только тогда когда, когда является фундаментальной каждая из числовых последовательностей $\{x_1^{(n)}\}$, $\{x_2^{(n)}\}$, \ldots, $\{x_m^{(n)}\}$ соответствующих координат точек $M_n$
    \end{lemma}
    \begin{proof}
        % TODO: AAAAAAAAAAAAAAAAAAAAAAAAAAAAAAAAAAAAAAAAAAAAAAAAAAAAAAAAAAAAAAAAAAAAAAAA
    \end{proof}
    % -----------------------------------------------------------------------------------



    \subsubsection{Теорема Больцано-Вейерштрасса для последовательностей точек пространства \texorpdfstring{$\mathbb{R}_m$}{Rm} (с доказательством).}
    \begin{theorem}[Больцано-Вейерштрасса]
        Из любой ограниченной последовательности $\{M_n\}$ точек $m$-мерного евклтдова пространства можно выделить сходящуюся подпоследовательность 
    \end{theorem}
    \begin{proof}
        % TODO: aaaaaa(454)
    \end{proof}
    % -----------------------------------------------------------------------------------



    \subsubsection{Определение предела функций многих переменных по Гейне и по Коши. Доказательство их эквивалентности.}
    \begin{definition}
        Число $b$ называется \textbf{пределом функции $f(x)$(по Гейне)} в точке $A$, если для любой сходящейся к $A$ последовательности $\{M_n\}$ точек множества $\{M\}$ задания этой функции, все элементы $M_n$ которой отличны от $A$, соответствующая числовая последовательность значений функции $\{f(M_n)\}$ cходится к числу $b$
    \end{definition}
    \begin{definition}
        Число $b$ называется \textbf{пределом функции $f(x)$(по Коши)} в точке $A$, если для любого положительного числа $\varepsilon$ найдется отвечающее ему положительное число $\delta$ такое, что для любой точки $M$ из множества $\{M\}$ задания этой функции, удовлетворяющей условию $0 < \rho(M, A) < \delta$, мправедливо неравенство $|f(M) - b| < \varepsilon$
    \end{definition}
    \begin{proof}
        % TODO: who?
    \end{proof}
    % -----------------------------------------------------------------------------------



    \subsubsection{Критерий Коши существования предела функции многих переменных в точке (с доказательством).}
    \begin{theorem}
        Для того чтобы функция $u = f(x)$  имела конечный предел в точке $M = A$, необходимо и достаточно чтобы она удовлетворяла в этой точке условию Коши
        \begin{remark}[Условие Коши]
            Функция $f(M)$ удовлетворяет условию Коши тогда и только тогда, когда $\forall \varepsilon>0 \exists \delta > 0: \forall M', M'' \in \{M\}\text{ множества задания функции}: 0 < \rho(M', A) < \delta, 0 < \rho(M'', A) < \delta$, справедливо неравенство $|f(M') - f(M'')| < \varepsilon$
        \end{remark}
    \end{theorem}
    \begin{proof}
        Идентично доказательству для аналога с одной переменной, просто вместо точки $a$ - $A$, а вместо $|x - a|$ - $\rho(M, A)$
    \end{proof}
    % -----------------------------------------------------------------------------------



    \subsubsection{Определение сложной функции многих переменных. Теорема о непрерывности сложной функции многих переменных (с доказательством).}
    \begin{definition}
        Пусть функции 
        \begin{gather*}
            x_i = \varphi(t_1, t_2, \ldots, t_k), i = \overline{1, m}
        \end{gather*}
        заданы на множестве $E$ евклидова пространства $\mathbb{R}^k$. Каждой точке $N(t_1, t_2, \ldots, t_k)$ множества $E$ они ставят в соответствие точку $M(x_1, x_2, \ldots, x_n)z$
    \end{definition}
    % -----------------------------------------------------------------------------------



    \subsubsection{Теорема о прохождении непрерывной функции многих переменных через промежуточное значение (с доказательством).}
    \subsubsection{Первая и вторая теоремы Вейерштрасса для функции многих переменных (с доказательствами).}
    \subsubsection{Определение равномерной непрерывности функции многих переменных. Теорема Кантора для функции многих переменных (с доказательством).}
    \subsection{Дифференциальное исчисление ФМП}
    \subsubsection{Определение дифференцируемости функции многих переменных в точке. Необходимое условие дифференцируемости, связь между дифференцируемостью и непрерывностью (с доказательствами).}
    \subsubsection{Достаточное условие дифференцируемости функции многих переменных в точке (с доказательством).}
    \subsubsection{Определение касательной плоскости к поверхности в трехмерном пространстве. Вывод уравнения касательной плоскости.}
    \subsubsection{Теорема о дифференцируемости сложной функции (с доказательством).}
    \subsubsection{Теорема: инвариантность формы первого дифференциала (с доказательством).}
    \subsubsection{Определение производной по направлению и градиента функции многих переменных. Геометрический смысл градиента (с доказательством).}
    \subsubsection{Определение частной производной высокого порядка. Теорема о достаточных условиях независимости смешанных производных от порядка дифференцирования (теорема Юнга, с доказательством).}
    \subsubsection{Определение частной производной высокого порядка. Теорема о достаточных условиях независимости смешанных производных от порядка дифференцирования (теорема Шварца, с доказательством).}
    \subsubsection{Формула Тейлора с остаточным членом в форме Лагранжа (с доказательством).}
    \subsubsection{Формула Тейлора с остаточным членом в форме Пеано (формулировка и доказательство леммы 1).}
    \subsubsection{Формула Тейлора с остаточным членом в форме Пеано (формулировка и доказательство леммы 2).}
    \subsubsection{Определение локального экстремума функции многих переменных. Необходимое условие экстремума функции многих переменных. (с доказательством).}
    \subsubsection{Определение локального экстремума функции многих переменных. Доказательство того, что если второй дифференциал функции в точке является знакопеременной КФ, то экстремума в данной точке нет.}
    \subsection{Функции заданные неявно. Экстремум}
    \subsubsection{Формулировка теоремы о существовании, непрерывности и дифференцируемости неявной функции, заданной одним уравнением. Доказательство существования и единственности.}
    \subsubsection{Формулировка теоремы о существовании, непрерывности и дифференцируемости неявной функции, заданной одним уравнением. Доказательство дифференцируемости.}
    \subsubsection{Теорема о разрешимости системы функциональных уравнений (с доказательством)}
    \subsubsection{Определение зависимости функций. Достаточные условия независимости функций.}
    \subsubsection{Определение зависимости функций. Теорема о функциональных матрицах.}
    \subsubsection{Определение условного локального экстремума функции многих переменных. Вывод необходимых условий экстремума (без функции Лагранжа).}
    \subsubsection{Определение условного локального экстремума функции многих переменных. Вывод необходимых условий экстремума в терминах множителей Лагранжа.}
    \subsubsection{Определение условного локального экстремума функции многих переменных. Вывод достаточных условий экстремума в терминах множителей Лагранжа.}
\end{document}