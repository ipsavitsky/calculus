\documentclass[10pt]{article}
\pdfoutput=1
\usepackage[T2A]{fontenc}
\usepackage[utf8]{inputenc}
\usepackage[english, russian]{babel}
\usepackage{NotesTeX_rus}

\begin{document}
    \tableofcontents
    \section{Определения}
    \begin{definition}
        Вектор $P = (x_0, \ldots, x_n)$ называется \textbf{разбиением(неразмеченным)} отрезка $[a, b]$, если $a = x_0 < x_1 < \ldots < x_n = b$, $i$-й отрезок разбиения обозначаем $\Delta_i := [x_{i-1}, x_i]$  
    \end{definition}
    \begin{definition}
        Разбиение $P'$ отрезка $[a, b]$ называется \textbf{измельчением} разбиения $P$ того же отрезка, если все элементы $P$ так же являются элементами $P'$
    \end{definition}
    \begin{definition}
        Разбиение $R$ называется \textbf{объединением разбиений} $P$ и $Q$ отрезка $[a, b]$, если $R$ включает в себя все элементы $P$ и $Q$ и только их. Обозначение: $R = P \cup Q$
    \end{definition}
    \begin{definition}
        Пусть функция $f$ определена на отрезке $[a, b]$, $P_{\xi}$ - его размеченное разбиение. Тогда число
        $$
        \sigma(f, P, \xi) = \sigma(f, P_{\xi}) = \sigma(P_{\xi}) := \sum_{i=1}^n f(\xi_i)\Delta x_i
        $$
        где $\Delta x_i = |\Delta_i| := x_i - x_{i-1}$ и $n = n(P)$, называется \textbf{интегральной суммой} функции $f$, соответствующей размеченному разбиению $P_{\xi}$
    \end{definition}
    \begin{definition}
        Пусть $P$ - разбиение отрезка $[a, b]$. Тогда числа
        \begin{gather*}
            S_{*}(P) = \underline{S}(P) := \sum_{i = 1}^{n(P)} m_i \Delta x_i \\
            S^{*}(P) = \overline{S}(P) := \sum_{i = 1}^{n(P)}M_i \Delta x_i\\
            \text{где }m_k := \inf_{x \in \Delta_k}f(x);\, M_k := \sup_{x \in \Delta_k}f(x)
        \end{gather*}
        называются \textbf{нижней} и \textbf{верхней суммами Дарбу} функции $F$ для разбиения $P$ 
    \end{definition}
    % TODO Определение предела интегральных сумм при стремлении диаметра разбиения к нулю на языке \varepsilon-\delta
    \section{Билеты}
    % -----------------------------------------------------------------------------------
    \subsection{Применение дифференциального исчисления для построения графиков}
    \subsubsection{Определение: строгого локального экстремума функции одной переменной. Теорема: первое достаточное условие экстремума (с доказательством).}
    \begin{definition}
        Пусть функция $f(x)$ определена всюду в некоторой окрестности точки $c$. Тогда эта функция имеет в точке с \textbf{локальный экстремум}, если существует такая окрестность точки $c$, что для всех точек этой окрестности значение $f(c)$ является наибольшим или наименьшим среди всех значений $f(x)$ этой функции
    \end{definition}
    \begin{theorem}[первое достаточное условие экстремума]
        Пусть $f(x)$ дифференцируема всюду в некоторой окрестности точки $c$, и пусть точка $c$ является стационарной точкой функции $f(x)$. Тогда если в пределах указанной окрестности производная $f(x)'$ положительна(отрицательна) слева от точки $c$ и отрицательна(положительна) справа от точки $c$, то функция $f(x)$ имеет в точке $c$ локальный максимум(минимум). Если же в пределах указанной окрестности точки $c$ производная $f(x)'$ имеет один и тот же знак слева и справа от точки $c$, то экстремума в точке $c$ нет
    \end{theorem}
    \begin{proof}
        \begin{enumerate}
            \item Пусть $f(x)'$ в пределах рассматриваемой окрестности положительна(отрицательна) слева и отрицательна(положительная) справа от $c$. Обозначим через $x_0$ любое значение аргумента из рассматриваемой окрестности, отличное от $c$. Достаточно доказать, что
            \begin{gather*}
                f(c)-f(x_0)>0\,(<0)
            \end{gather*}
            Т.к. $f(x)$ дифференцируема всюду на $U(c)$, то она дифференцируема на сегменте, ограниченном $c$ и $x_0 \Rightarrow$ выполнены все условия теоремы Лагранжа и
            \begin{gather}
                f(c) - f(x_0) = f'(\xi)(c-x_0)\label{eq:lagrange}
            \end{gather}
            где $\xi$ - некоторое значение аргумента между $c$ и $x_0$. Поскольку производная $f'(\xi)$ положительна(отрицательна) при $x_0 < c$ и отрицательна(положительна) при $x_0 > c$, правая часть \ref{eq:lagrange} положительна(отрицательна)
            \item Пусть теперь производная $f'(x)$ имеет один и тот же знак слева и справа от $c$. Обозначая через $x_0$ любое значение аргумента, отличное от $c$ и повторяя рассуждения выше, мы докажем, что правая часть \ref{eq:lagrange} имеет разные знаки при $x_0 < c(x_0 > c) \Rightarrow$ в $c$ нет экстремума.
        \end{enumerate}
    \end{proof}
    % -----------------------------------------------------------------------------------



    \subsubsection{Определение: строгого локального экстремума функции одной переменной. Теорема: второе достаточное условие экстремума (с доказательством).}
    \begin{definition}
        Пусть функция $f(x)$ определена всюду в некоторой окрестности точки $c$. Тогда эта функция имеет в точке с \textbf{локальный экстремум}, если существует такая окрестность точки $c$, что для всех точек этой окрестности значение $f(c)$ является наибольшим или наименьшим среди всех значений $f(x)$ этой функции
    \end{definition}
    \begin{theorem}[второе достаточное условие экстремума]
        Пусть $f(x)$ имеет в данной стационарной точке $c$ конечную вторую производную. Тогда функция $f(x)$ имеет в точке $c$ локальный максимум(минимум), если $f^{(2)} < 0\,(>0)$ 
    \end{theorem}
    \begin{proof}
        Из условия $f^{(2)} < 0\,(>0)$ и из доказанной в гл.6 теоремы 6.1 вытекает, что функция $f'(x)$ убывает(возрастает) в точке $c$. Поскольку по условию $f'(c) = 0$, то найдется такая окрестность точки $c$, в пределах которой $f'(x)$ положительна(отрицательна) слева от $c$ и отрицательна(положительна) справа от $c$. Но тогда по первому достаточну условию экстремума $f(x)$ имеет в точке $c$ локальный максимум(минимум)
        % TODO: настроить ссылку на теорему 6.1
    \end{proof}
    % -----------------------------------------------------------------------------------



    \subsubsection{Определение: строгого локального экстремума функции одной переменной. Теорема: третье достаточное условие экстремума (с доказательством).}
    \begin{definition}
        Пусть функция $f(x)$ определена всюду в некоторой окрестности точки $c$. Тогда эта функция имеет в точке с \textbf{локальный экстремум}, если существует такая окрестность точки $c$, что для всех точек этой окрестности значение $f(c)$ является наибольшим или наименьшим среди всех значений $f(x)$ этой функции
    \end{definition}
    \begin{theorem}[третье достаточное условие экстремума]
        Пусть $n \geq 1$ - некоторое неяетное число , и пусть функция $y = f(x)$ имеет производную порядка $n$ в некоторой окрестности точки $c$ и производную порядка $n+1$ в самой точке $c$. Тогда, если
        \begin{gather}
            f'(c) = f^{(2)}(c) = \ldots = f^{(n)}(c) = 0,\, f^{(n+1)} \neq 0 \label{eq:extr_cond}
        \end{gather}
        то $y = f(x)$ имеет в точке $c$ локальный максимум(минимум) при $f^{(n+1)} < 0\,(>0)$
    \end{theorem}
    \begin{proof}
        При $n = 1$ теорема совпадает со вторым достаточным условием экстремума, так что достаточно доказать для нечетного $n \geq 3$ \\
        Пусть $n \geq 3;\; f^{(n+1)} > 0$. Докажем, что $y = f(x)$ имеет в $c$ локальный минимум. Доказательства для остальных случаев проводятся аналогично\\
        Так как $f^{(n+1)} > 0$, то, в силу теоремы 6.1 о достаточном условии возрастании функции в точке, функция $f^{(n)}(x)$ отрицательная слева от $c$ и положительна справа от $c$. Тогда разложим $f(x)$ в окрестности $c$ с остаточным членом в форме Лагранжа. Тогда для всех $x$ из достаточно малой окрестности $c$ между $x$ и $c$ найдется $\xi$ такая, что
        \begin{gather*}
            f'(x) = f'(c)+\frac{f^{(2)}(c)}{1!}(x-c) + \ldots + \frac{f^{(n-1)}(c)}{(n-2)!}(x-c)^{n-2} + \frac{f^{(n)}(\xi)}{(n-1)!}(x-c)^{n-1}
        \end{gather*}
        В силу соотношений \ref{eq:extr_cond} написанное разложение принимает вид
        \begin{gather}
            f'(x) = \frac{f^{(n)}(\xi)}{(n-1)!} (x-c)^{n-1} \label{eq:taylor_simp}
        \end{gather}
        Так как $\xi$ лежит между $x$ и $c$, то для всех $x$ из достаточно малой окрестности точки $c$ велчичина $f^{(n)}(\xi)$(а значит, так как $n$ нечетная - и вся правая часть \ref{eq:taylor_simp}) отрицательная слева от $c$ и положительна справа от $c$\\
        В силу первого достаточного условия экстремума $y = f(x)$ имеет в $c$ локальный минимум 
        % TODO: ссылка на теорему 6.1
    \end{proof}
    % -----------------------------------------------------------------------------------



    \subsubsection{Определение: график функции на данном интервале имеет выпуклость, направленную вверх (вниз). Теорема о связи между знаком второй производной и выпуклостью (с доказательством).}
    \begin{definition}
        График функции $y = f(x)$ имеет на интервале $(a, b)$ \textbf{выпуклость, направленную вверх(вниз)}, если график этой функции в пределах указанного интервала лежит не выше(не ниже) любой своей касательной
    \end{definition}
    \begin{remark}
        Термин "график лежит не выше(не ниже) своей касательной" имеет смысл, т.к. касательная не параллельна $Ox$
    \end{remark}
    \begin{theorem}[о связи между знаком второй производной и выпуклостью]
        Если функция $y = f(x)$ имеет на интервале $(a, b)$ конечную вторую производную и если эта производная неотрицательна(неположительна) всюду на этом интервале, то график функции $y = f(x)$ имеет на интервале $(a, b)$ выпуклость, направленную вниз(вверх)
    \end{theorem}
    \begin{proof}
        Ради определнности рассмотрим случай, когда вторая производная $f^{(2)} \geq 0$ всюду на $(a, b)$. Обозначим через $c$ любую точку интервала $(a, b)$. Требуется доказать, что график функции $y = f(x)$ в пределах интервала $(a, b)$ лежит не ниже касательной, проходящей через точку $M(c, f(c))$. Запишем уравнение указанной касательной, обоззначая ее текущую ординату через $Y$. Поскольку угловой коэфициент указанной касательной равен $f'(c)$, то ее уравнение имеет вид
        \begin{gather}
            Y - f(c) = f'(c)(x-c) \label{eq:angle_coef}
        \end{gather}
        Разложим функцию $f(x)$ в окрестности точки $c$ по формули Тейлора, беря в этой формуле $n-1$. Получим
        \begin{gather}
            y = f(x) = f(c) + \frac{f'(c)}{1!}(x-c) + \frac{f^{(2)}(\xi)}{2!}(x-c)^2 \label{eq:taylor_n-1}
        \end{gather}
        где остаточный член взят в форме Лагранжа, $\xi$ заключена между $c$ и $x$\\
        Сопостовляя \ref{eq:taylor_n-1} и \ref{eq:angle_coef} будем иметь
        \begin{gather}
            y-Y=\frac{f^{(2)}(\xi)}{2}(x-c)^2 \label{eq:convex_eq}
        \end{gather}
        Поскольку вторая производная по условию неотрицательна всюду на $(a, b)$, то правая часть \ref{eq:convex_eq} неотрицательна, т.е. для всех $x$ из $(a, b)$ лежит не ниже касательной \ref{eq:angle_coef}\\
        Аналогично для $f^{(2)}(x) \leq 0$
    \end{proof}
    % -----------------------------------------------------------------------------------



    \subsubsection{Определение точки перегиба графика функции. Лемма о взаимном расположении графика функции и касательной к нему, проведенной в точке перегиба (с доказательством).}
    \begin{definition}
        Точка $M(c, f(c))$ графика функции $y = f(x)$ называется \textbf{точкой перегиба} этого графика, если существует такая окрестность точки $c$ оси абсцисс, в пределах которой график функции $y = f(x)$ слева и справа от $c$ имеет разные направления выпуклости.
    \end{definition}
    \begin{lemma}
        Пусть функция $y = f(x)$ имеет производную $f'(x)$ всюду в $\delta$-окрестности точки $c$. Тогда, если график функции $y = f(x)$ имеет на интервале $(c, c+\delta)$ выпуклость, направленную вниз(вверх), то всюду в пределах интервала $(c, x+\delta)$ этот график лежит не ниже(не выше) касательной к графику, проведенной в точке $M(c, f(c))$
    \end{lemma}
    \begin{proof}
        Рассмотрим последовательность $\{x_n\}$ точек интервала $(c, c+\delta)$, сходящуюся к точке $c$. Через каждую точку $M_n(x_n, f(x_n))$ графика функции $y = f(x)$ проведем касательную к этому графику, т.е. прямую
        \begin{gather*}
            Y_n = f(x_n) + f'(x_n)(x - x_n)
        \end{gather*}
        Так как по условию график функции $y = f(x)$ имеет на интервале $(c, c+\delta)$ выпуклость, направленную вниз(вверх), то для любого $n$ и любой фиксированной точки $x$ интервала $(c, c+\delta)$
        \begin{gather}
            f(x) - Y_n=f(x)-f(x_n) - f'(x_n)(x-x_n) \geq 0\; (\leq 0) \label{ineq: convex}
        \end{gather}
        Из условия непрерывности $f'(x)$(и тем более $f(x)$) в точке $c$ и из определения непрерывности по Гейне вытекает, что существует предел
        \begin{gather*}
            \lim_{n \to \infty}(f(x) - Y_n) = \lim_{n \to \infty}\{f(x) - f(x_n) - f'(x_n)(x-x_n)\} = f(x) - f(c) - f'(c)(x-c)
        \end{gather*}
        Из существования последнего предела в силу неравенства \ref{ineq: convex} и теоремы 3.13 мы получим, что
        \begin{gather*}
            f(x)-f(c)-f'(c)(x-c) \geq 0\; (\leq 0)
        \end{gather*}
        Если обозначить через $Y$ текущую ординату касательной \ref{eq:angle_coef}, проходящей через точку $M(c, f(c))$, то последнее неравенство можно переписать в виде
        \begin{gather*}
            f(x) - Y \geq 0\; (\leq 0)
        \end{gather*}
        Итак, переходя \ref{ineq: convex} к пределу при $n \to \infty$ и используя теорему 3.13, мы получим, что $f(x) - Y \geq 0\, (\leq 0)$ для любой фиксированной точки $x$ из интервала $(c, c+\delta)$, причем $Y$ обозначает текущую ординату касатеьной, проведенной через точку $M(c, f(c))$
        % TODO: настроить ссылку на 3.13
        \end{proof}
        \begin{remark}
            Аналогично формулируется и доказывается лемма 1 и для случая когда график функции имеет определенное направление выпоклости на интервале $(c, c+\delta)$, а на интервале $(c-\delta, c)$
        \end{remark}
        \begin{lemma}
            % TODO: Лемма 2?
        \end{lemma}
    % -----------------------------------------------------------------------------------



    \subsubsection{Определение точки перегиба графика функции. Теорема: необходимое условие перегиба(с доказательством).}
    \begin{definition}
        Точка $M(c, f(c))$ графика функции $y = f(x)$ называется \textbf{точкой перегиба} этого графика, если существует такая окрестность точки $c$ оси абсцисс, в пределах которой график функции $y = f(x)$ слева и справа от $c$ имеет разные направления выпуклости.
    \end{definition}
    \begin{theorem}[необходимое условие перегиба(дважды дифференцируемой функции)]
        Если функция $y = f(x)$ имеет в точке $c$ вторую производную и график этой функции имеет перегиб в точке $M(c, f(c))$, то $f^{(2)}(c) = 0$
    \end{theorem}
    \begin{proof}
        Пусть $Y$ - текущая ордината касательной $Y=f(c)+f'(c)(x-c)$, проходящей через точку графика $M(c, f(c))$\\
        Рассмотрим функцию:
        \begin{gather*}
            F(x)=f(x) - Y = f(x) - f(c) - f'(c)(x-c)
        \end{gather*}
        равную разности $f(x)$ и линейной функции $f(c)+f'(c)(x-c)$\\
        $F(x)$, как и $f(x)$, имеет в точке $с$ вторую производную. В силу леммы 2 в малой окрестности точки $c$ график функции $y = f(x)$ лежит слева и справа от $c$ по разные стороны от касательной, проходящей через точку $M(c, f(c))$, а потому функция $F(x)$ в малой окрестности точки $c$ имеет слева и справа от $c$ разные знаки.\\
        Значит, $F(x)$ не может иметь в точке $c$ локального экстремума\\
        Предположим, что $f^{(2)} \neq 0$. Тогда, поскольку $F'(x) = f'(x) - f'(c),\, F^{(2)}(x) = f^{(2)}(2)$
    \end{proof}
    % -----------------------------------------------------------------------------------



    \subsubsection{Определение точки перегиба графика функции. Теорема: первое достаточное условие перегиба (с доказательством).}
    \begin{definition}
        Точка $M(c, f(c))$ графика функции $y = f(x)$ называется \textbf{точкой перегиба} этого графика, если существует такая окрестность точки $c$ оси абсцисс, в пределах которой график функции $y = f(x)$ слева и справа от $c$ имеет разные направления выпуклости.
    \end{definition}
    % -----------------------------------------------------------------------------------



    \subsubsection{Определение точки перегиба графика функции. Теорема: второе достаточное условие перегиба (с доказательством).}
    \begin{definition}
        Точка $M(c, f(c))$ графика функции $y = f(x)$ называется \textbf{точкой перегиба} этого графика, если существует такая окрестность точки $c$ оси абсцисс, в пределах которой график функции $y = f(x)$ слева и справа от $c$ имеет разные направления выпуклости.
    \end{definition}
    % -----------------------------------------------------------------------------------



    \subsubsection{Определение точки перегиба графика функции. Теорема: третье достаточное условие перегиба (с доказательством).}
    \begin{definition}
        Точка $M(c, f(c))$ графика функции $y = f(x)$ называется \textbf{точкой перегиба} этого графика, если существует такая окрестность точки $c$ оси абсцисс, в пределах которой график функции $y = f(x)$ слева и справа от $c$ имеет разные направления выпуклости.
    \end{definition}
    % -----------------------------------------------------------------------------------



    \subsubsection{Определение вертикальной и наклонной асимптот к графику функции. Теорема: правила нахождения наклонной асимптоты (с доказательством).}
    % -----------------------------------------------------------------------------------


    
    \subsection{Определенный интеграл}
    \subsubsection{Определение интегрируемости функции по Риману. Теорема: необходимое условие интегрируемости (с доказательством).}
    \subsubsection{Определение интегральной суммы, верхней и нижней сумм Дарбу. Леммы о соотношениях между ними (с доказательствами).}
    \subsubsection{Определение разбиения, диаметра разбиения. Лемма об изменении верхней суммы Дарбу при добавлении к разбиению новых точек (с доказательством).}
    \subsubsection{Определение верхнего и нижнего интегралов Дарбу. Леммы о соотношении между верхней и нижней суммами Дарбу, соответствующими различным разбиениям, и о соотношении между суммами Дарбу и интегралами Дарбу (с доказательствами).}
    \subsubsection{Основная лемма Дарбу (с доказательством).}
    \subsubsection{Критерий интегрируемости функции по Риману на отрезке (в терминах верхней и нижней сумм Дарбу, с доказательством).}
    \subsubsection{Теорема об интегрируемости непрерывной функции на отрезке (с доказательством).}
    \subsubsection{Теорема об интегрируемости на отрезке функции, имеющей разрывы (с доказательством).}
    \subsubsection{Теорема об интегрируемости функции, монотонной на отрезке (с доказательством).}
    \subsubsection{Теорема об интегрируемости композиции функций на отрезке (с доказательством).}
    \subsubsection{Свойства определенного интеграла: интегрируемость суммы, разности, произведения функции и константы, произведения функций (с доказательством).}
    \subsubsection{Свойства определенного интеграла: интегрируемость на подотрезке, аддитивность (с доказательством).}
    \subsubsection{Интегрирование неравенств, оценка модуля от интеграла (с доказательством).}
    \subsubsection{Первая теорема о среднем значении интеграла (с доказательством).}
    \subsubsection{Определение интеграла с переменным верхним пределом. Теорема: непрерывность интеграла с переменным верхним пределом (с доказательством).}
    \subsubsection{Определение интеграла с переменным верхним пределом. Теорема: дифференцируемость интеграла с переменным верхним пределом (с доказательством).}
    \subsubsection{Определение интеграла с переменным верхним пределом. Теорема: формула Ньютона-Лейбница (с доказательством).}
    \subsubsection{Формулы замены переменной и интегрирования по частям в определенном интеграле (с доказательством).}
    \subsection{Геометрические приложения определенного интеграла}
    \subsubsection{Определение длины плоской кривой. Теорема о спрямляемости и длине дуги кривой, заданной параметрически (с доказательством).}
    \subsubsection{Определение площади плоской фигуры. Два критерия квадрируемости (через приближение простейшими и квадрируемыми (с доказательством).}
    \subsubsection{Определение площади плоской фигуры. Теорема о квадрируемости и площади криволинейной трапеции (с доказательством).}
    \subsubsection{Определение площади плоской фигуры. Теорема о квадрируемости и площади криволинейного сектора (с доказательством).}
    \subsubsection{Определение объема тела в пространстве. Два критерия кубируемости (через приближение простейшими и кубируемыми, с доказательством).}
    \subsubsection{Определение объема тела в пространстве. Кубируемость цилиндрического тела и ступенчатого тела (с доказательством).}
    \subsubsection{Определение объема тела в пространстве. Кубируемость тела, образованного вращением криволинейной трапеции вокруг оси Ox (с доказательством).}
    \subsection{Несобственные интегралы}
    \subsubsection{Определение несобственного интеграла первого рода. Теоремы: замена переменной и интегрирование по частям в несобственном интеграле первого рода (с доказательствами).}
    \subsubsection{Критерий Коши и признак сравнения сходимости несобственного интеграла первого рода (с доказательствами).}
    \subsubsection{Определение абсолютной и условной сходимости несобственного интеграла первого рода. Признак Дирихле-Абеля сходимости несобственного интеграла первого рода (с доказательством).}
    \subsubsection{Определение несобственного интеграла первого и второго рода. Понятие главного значения несобственного интеграла первого и второго рода. Главное значение интеграла первого рода для нечетной функции.}
    \subsubsection{Метод хорд нахождения корня уравнения и его обоснование.}
    \subsubsection{Метод касательных нахождения корня уравнения и его обоснование.}
    \subsubsection{Метод прямоугольников вычисления определенного интеграла (с выводом оценки погрешности).}
    \subsubsection{Формула Тейлора с остаточным членом в интегральной форме (с доказательством).}
    \subsection{Функции многих переменных}
    \subsubsection{Определение предела последовательности в пространстве $\mathbb{R}_m$. Леммы об эквивалентности сходимости последовательности в пространстве $\mathbb{R}_m$ и покоординатной сходимости, фундаментальности в пространстве $\mathbb{R}_m$ и покоординатной фундаментальности (с доказательствами).}
    \subsubsection{Теорема Больцано-Вейерштрасса для последовательностей точек пространства $\mathbb{R}_m$ (с доказательством).}
    \subsubsection{Определение предела функций многих переменных по Гейне и по Коши. Доказательство их эквивалентности.}
    \subsubsection{Критерий Коши существования предела функции многих переменных в точке (с доказательством).}
    \subsubsection{Определение сложной функции многих переменных. Теорема о непрерывности сложной функции многих переменных (с доказательством).}
    \subsubsection{Теорема о прохождении непрерывной функции многих переменных через промежуточное значение (с доказательством).}
    \subsubsection{Первая и вторая теоремы Вейерштрасса для функции многих переменных (с доказательствами).}
    \subsubsection{Определение равномерной непрерывности функции многих переменных. Теорема Кантора для функции многих переменных (с доказательством).}
    \subsection{Дифференциальное исчисление ФМП}
    \subsubsection{Определение дифференцируемости функции многих переменных в точке. Необходимое условие дифференцируемости, связь между дифференцируемостью и непрерывностью (с доказательствами).}
    \subsubsection{Достаточное условие дифференцируемости функции многих переменных в точке (с доказательством).}
    \subsubsection{Определение касательной плоскости к поверхности в трехмерном пространстве. Вывод уравнения касательной плоскости.}
    \subsubsection{Теорема о дифференцируемости сложной функции (с доказательством).}
    \subsubsection{Теорема: инвариантность формы первого дифференциала (с доказательством).}
    \subsubsection{Определение производной по направлению и градиента функции многих переменных. Геометрический смысл градиента (с доказательством).}
    \subsubsection{Определение частной производной высокого порядка. Теорема о достаточных условиях независимости смешанных производных от порядка дифференцирования (теорема Юнга, с доказательством).}
    \subsubsection{Определение частной производной высокого порядка. Теорема о достаточных условиях независимости смешанных производных от порядка дифференцирования (теорема Шварца, с доказательством).}
    \subsubsection{Формула Тейлора с остаточным членом в форме Лагранжа (с доказательством).}
    \subsubsection{Формула Тейлора с остаточным членом в форме Пеано (формулировка и доказательство леммы 1).}
    \subsubsection{Формула Тейлора с остаточным членом в форме Пеано (формулировка и доказательство леммы 2).}
    \subsubsection{Определение локального экстремума функции многих переменных. Необходимое условие экстремума функции многих переменных. (с доказательством).}
    \subsubsection{Определение локального экстремума функции многих переменных. Доказательство того, что если второй дифференциал функции в точке является знакопеременной КФ, то экстремума в данной точке нет.}
    \subsection{Функции заданные неявно. Экстремум}
    \subsubsection{Формулировка теоремы о существовании, непрерывности и дифференцируемости неявной функции, заданной одним уравнением. Доказательство существования и единственности.}
    \subsubsection{Формулировка теоремы о существовании, непрерывности и дифференцируемости неявной функции, заданной одним уравнением. Доказательство дифференцируемости.}
    \subsubsection{Теорема о разрешимости системы функциональных уравнений (с доказательством)}
    \subsubsection{Определение зависимости функций. Достаточные условия независимости функций.}
    \subsubsection{Определение зависимости функций. Теорема о функциональных матрицах.}
    \subsubsection{Определение условного локального экстремума функции многих переменных. Вывод необходимых условий экстремума (без функции Лагранжа).}
    \subsubsection{Определение условного локального экстремума функции многих переменных. Вывод необходимых условий экстремума в терминах множителей Лагранжа.}
    \subsubsection{Определение условного локального экстремума функции многих переменных. Вывод достаточных условий экстремума в терминах множителей Лагранжа.}
\end{document}